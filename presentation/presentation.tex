\documentclass[aspectratio=169]{beamer}

% ============================================================
% THEME UND PACKAGES
% ============================================================
\usetheme{metropolis}
\usepackage[utf8]{inputenc}
\usepackage[T1]{fontenc}
\usepackage[ngerman]{babel}
\usepackage{graphicx}
\usepackage{booktabs}
\usepackage{xcolor}
\usepackage{tikz}
\usepackage{pgfplots}
\pgfplotsset{compat=1.18}
\usepackage[edges]{forest}
\usetikzlibrary{shapes,arrows,positioning,fit,backgrounds,calc}
\usepackage{tcolorbox}
\tcbuselibrary{skins,breakable}
\usepackage{setspace}
\usepackage{microtype}

% ============================================================
% FARBSCHEMA
% ============================================================
\definecolor{primarydark}{RGB}{15,23,42}
\definecolor{primaryblue}{RGB}{30,64,175}
\definecolor{accentcyan}{RGB}{6,182,212}
\definecolor{accentorange}{RGB}{251,146,60}
\definecolor{accentred}{RGB}{239,68,68}
\definecolor{accentgreen}{RGB}{34,197,94}
\definecolor{accentpurple}{RGB}{168,85,247}
\definecolor{lightgray}{RGB}{241,245,249}
\definecolor{textgray}{RGB}{71,85,105}

% Farbschema
\definecolor{darkaccent}{RGB}{35,55,75}
\definecolor{chartcolor1}{RGB}{231,76,60}
\definecolor{chartcolor2}{RGB}{52,152,219}
\definecolor{chartcolor3}{RGB}{46,204,113}
\definecolor{chartcolor4}{RGB}{241,196,15}
\definecolor{chartcolor5}{RGB}{155,89,182}
\definecolor{chartcolor6}{RGB}{230,126,34}

\definecolor{rustcolor}{RGB}{183,65,14}

% ============================================================
% BEAMER THEME KONFIGURATION
% ============================================================
\setbeamercolor{normal text}{fg=primarydark,bg=white}
\setbeamercolor{alerted text}{fg=accentred}
\setbeamercolor{example text}{fg=accentgreen}
\setbeamercolor{frametitle}{bg=primarydark,fg=white}
\setbeamercolor{title separator}{fg=accentcyan}
\setbeamercolor{progress bar}{fg=accentcyan,bg=primarydark!20}
\setbeamercolor{block title}{fg=white,bg=primaryblue}
\setbeamercolor{block body}{fg=primarydark,bg=lightgray}
\setbeamercolor{block title example}{fg=white, bg=rustcolor}
\setbeamercolor{block body example}{bg=rustcolor!8}
\setbeamercolor{itemize item}{fg=primaryblue}
\setbeamercolor{enumerate item}{fg=primaryblue}

\metroset{
  titleformat=regular,
  sectionpage=progressbar,
  numbering=fraction,
  progressbar=frametitle,
  block=fill
}

% ============================================================
% CUSTOM BOXES
% ============================================================
\newtcolorbox{demobox}[1]{
  enhanced,
  colback=white,
  colframe=primaryblue,
  colbacktitle=primaryblue,
  coltitle=white,
  coltext=primarydark,
  title={\textbf{#1}},
  fonttitle=\sffamily\bfseries\small,
  fontupper=\sffamily\footnotesize,
  arc=3pt,
  boxrule=0pt,
  leftrule=3pt,
  toptitle=4pt,
  bottomtitle=4pt,
  left=8pt, right=8pt, top=4pt, bottom=6pt,
  before skip=6pt,
  after skip=6pt
}

\newtcolorbox{infobox}[1]{
  enhanced,
  colback=accentcyan!5,
  colframe=accentcyan,
  coltext=primarydark,
  title={\textbf{#1}},
  coltitle=white,
  colbacktitle=accentcyan,
  fonttitle=\sffamily\bfseries\small,
  fontupper=\sffamily\footnotesize,
  arc=3pt,
  boxrule=0pt,
  leftrule=3pt,
  toptitle=5pt,
  bottomtitle=5pt,
  left=8pt, right=8pt, top=4pt, bottom=6pt,
  before skip=6pt,
  after skip=6pt
}

\newtcolorbox{warnbox}[1]{
  enhanced,
  colback=accentorange!8,
  colframe=accentorange,
  coltext=primarydark,
  title={\textbf{#1}},
  coltitle=accentorange!80!black,
  colbacktitle=accentorange!25,
  fonttitle=\sffamily\bfseries\small,
  fontupper=\sffamily\footnotesize,
  arc=3pt,
  boxrule=0pt,
  leftrule=3pt,
  toptitle=5pt,
  bottomtitle=5pt,
  left=8pt, right=8pt, top=4pt, bottom=6pt,
  before skip=6pt,
  after skip=6pt
}

% ============================================================
% TYPOGRAPHY
% ============================================================
\setbeamerfont{title}{size=\LARGE,series=\bfseries}
\setbeamerfont{subtitle}{size=\normalsize}
\setbeamerfont{frametitle}{size=\large,series=\bfseries}
\setbeamerfont{block title}{size=\small,series=\bfseries}

\setlength{\leftmargini}{1em}
\setlength{\leftmarginii}{0.8em}

\newcommand{\code}[1]{\texttt{\textcolor{primaryblue}{#1}}}

% ============================================================
% TITELSEITE
% ============================================================
\title{Ransomware}
\date{23.01.2026}
\institute{Hochschule für Technik Stuttgart – Aktuelle Themen der IT-Sicherheit}

\begin{document}

% ============================================================
% TITELFOLIE
% ============================================================
{
\setbeamertemplate{background}{
  \begin{tikzpicture}[overlay,remember picture]
    \shade[top color=primarydark,bottom color=primaryblue!50!primarydark] 
      (current page.south west) rectangle (current page.north east);
    \draw[accentcyan,line width=2pt] 
      ([yshift=-0.5cm]current page.center) ++(-5,0) -- ++(10,0);
  \end{tikzpicture}
}
\setbeamercolor{title}{fg=white}
\setbeamercolor{subtitle}{fg=accentcyan}
\setbeamercolor{author}{fg=white!80}
\setbeamercolor{institute}{fg=white!60}
\setbeamercolor{date}{fg=white!60}
\begin{frame}[plain]
  \vspace{2cm}
  \begin{center}
    {\usebeamerfont{title}\usebeamercolor[fg]{title}\inserttitle}\\[1.5em]
    {\footnotesize\usebeamercolor[fg]{institute}\insertinstitute}\\[0.3em]
    {\footnotesize\usebeamercolor[fg]{date}\insertdate}
  \end{center}
\end{frame}
}

% ============================================================
% AGENDA
% ============================================================
\begin{frame}{Agenda}
  \tableofcontents
\end{frame}

% ============================================================
% SEKTION 1: AKTUELLE BEDROHUNGSLAGE
% ============================================================
\section{Aktuelle Bedrohungslage}

\begin{frame}{Finanzielle Auswirkungen}
  \begin{columns}[T]
    \begin{column}{0.35\textwidth}
      \textbf{Geschätzte Weltweite Schäden}
      \vspace{0.5em}
      
      \begin{tabular}{ll}
        2017: & \textbf{5 Mrd. USD}\\
        2024: & \textbf{42 Mrd. USD}\\
      \end{tabular}
      
      \vspace{0.5em}
      \textcolor{accentred}{\textbf{+740\%}} Anstieg in 7 Jahren
    \end{column}
    \begin{column}{0.62\textwidth}
      \begin{tikzpicture}
        \begin{axis}[
          width=\textwidth,
          height=0.6\textheight,
          xlabel={\footnotesize Jahr},
          ylabel={\footnotesize Schaden (Mrd. USD)},
          ymin=0, ymax=48,
          xtick={2015,2017,2019,2021,2024},
          ytick={0,10,20,30,40},
          grid=both,
          grid style={line width=.1pt, draw=lightgray},
          x tick label style={/pgf/number format/1000 sep={},font=\scriptsize},
          y tick label style={font=\scriptsize},
        ]
        \addplot[fill=chartcolor1!30,draw=chartcolor1,line width=1.5pt,mark=*,mark options={fill=chartcolor1,scale=0.8}] coordinates {
          (2015,0.325)(2017,5)(2019,11.5)(2021,20)(2024,42)
        } \closedcycle;
        \end{axis}
      \end{tikzpicture}
    \end{column}
  \end{columns}
\end{frame}

\begin{frame}{ENISA Threat Landscape 2024}
  \begin{columns}[T]
    \begin{column}{0.35\textwidth}
      \textbf{Top-Bedrohungen EU:}
      \begin{enumerate}
        \item DDoS/RDoS (46,31\%)
        \item \alert{Ransomware (27,33\%)}
        \item Data Breaches (15,87\%)
        \item Social Engineering (5,37\%)
        \item Malware (2,45\%)
        \item Zero-Day (0,11\%)
      \end{enumerate}
      
      \vspace{0.3em}
      \small\textbf{Ransomware: Zweitgrößte Bedrohung}
    \end{column}
    \begin{column}{0.62\textwidth}
      \begin{tikzpicture}
        \begin{axis}[
          width=\textwidth,
          height=0.65\textheight,
          xbar,
          xlabel={\footnotesize Vorfälle (Tausend)},
          symbolic y coords={Zero-Day, Malware, Social Eng., Data, Ransomware, DDoS},
          ytick=data,
          nodes near coords,
          nodes near coords style={font=\scriptsize},
          xmin=0, xmax=2.8,
          bar width=0.4cm,
          y tick label style={font=\scriptsize},
          x tick label style={font=\scriptsize},
          enlarge y limits=0.12,
        ]
        \addplot[fill=chartcolor1,draw=chartcolor1!80!black] coordinates {
          (0.01,Zero-Day)(0.13,Malware)(0.29,Social Eng.)(0.84,Data)(1.45,Ransomware)(2.46,DDoS)
        };
        \end{axis}
      \end{tikzpicture}
    \end{column}
  \end{columns}
\end{frame}

% ============================================================
% SEKTION 2: ZIELE & MOTIVATION
% ============================================================
\section{Ziele \& Motivation}

\begin{frame}{Motivation der Angreifer}
  \textbf{Warum diese Ziele?}
  \vspace{0.5em}
  
  \begin{columns}[T]
    \begin{column}{0.48\textwidth}
      \begin{block}{Finanzielle Motive}
        \small Zahlungsbereitschaft bei kritischen Systemen
      \end{block}
      \vspace{0.3em}
      \begin{block}{Kritische Infrastrukturen}
        \small Hoher Druck durch Ausfallkosten
      \end{block}
    \end{column}
    \begin{column}{0.48\textwidth}
      \begin{block}{Sensible Daten}
        \small Erpressungspotenzial durch Datenleaks
      \end{block}
      \vspace{0.3em}
      \begin{block}{Geopolitische Faktoren}
        \small Destabilisierung und Spionage
      \end{block}
    \end{column}
  \end{columns}
  
  \vspace{0.5em}
  \begin{center}
    \alert{\textbf{Zunehmende Professionalisierung und Organisation}}
  \end{center}
\end{frame}

% ============================================================
% RANSOMWARE AS A BUSINESS
% ============================================================

\begin{frame}{Warum die Bedrohung bleibt: Die Ökonomie}
  \textbf{Ransomware-as-a-Service (RaaS)} ist eine hochprofessionelle Schattenwirtschaft mit Milliardenumsätzen:
  \vspace{0.5em}

  \begin{columns}[T]
    \begin{column}{0.32\textwidth}
      \begin{block}{1. Access Provider}
        \footnotesize
        Spezialisten, die Zugänge zu Netzwerken potenzieller Opfer verkaufen.
      \end{block}
    \end{column}
    \begin{column}{0.32\textwidth}
      \begin{block}{2. RaaS Provider}
        \footnotesize
        Entwickler der Malware. Bieten C2-Infrastruktur und Support gegen Provision.
      \end{block}
    \end{column}
    \begin{column}{0.32\textwidth}
      \begin{block}{3. Affiliates}
        \footnotesize
        Kaufen Zugänge, mieten Malware und führen den Angriff durch.
      \end{block}
    \end{column}
  \end{columns}
\end{frame}

\begin{frame}{Professionalisierungsgrad: Beispiel LockBit}
  Die Qualität der Services übertrifft teilweise legale SaaS-Anbieter.
  
  \vspace{0.5em}
  
  \begin{demobox}{Beispiel: LockBit 3.0 Bug-Bounty}
    LockBit hat ein eigenes Prämienprogramm für Hinweise zur Verbesserung ihrer Schadsoftware aufgelegt:
    \vspace{0.2em}
    
    \textit{"Locker Bugs: Any errors during encryption ... that lead to corrupted files or to the possibility of decrypting files without getting a decryptor."}
  \end{demobox}

  \vspace{0.3em}
  
  \begin{warnbox}{Wirtschaftliches Risiko}
    Im Gegensatz zu Banken ("Too big to fail") gilt für Ransomware-Gruppen: \textbf{"Too big to prevail"}. Wer zu groß wird, zieht zu viel Aufmerksamkeit der Strafverfolger auf sich.
  \end{warnbox}
\end{frame}

% ============================================================
% SEKTION 3: ARTEN VON RANSOMWARE
% ============================================================
\section{Arten von Ransomware}

\begin{frame}{Grundlegende Typen}
  \begin{columns}[T]
    \begin{column}{0.48\textwidth}
      \begin{block}{1. Crypto-Ransomware}
        \small
        \begin{itemize}
          \item Verschlüsselt Benutzerdateien
          \item Verwendet AES, RSA
          \item System bleibt funktional
          \item Wiederherstellung oft unmöglich
        \end{itemize}
      \end{block}
    \end{column}
    \begin{column}{0.48\textwidth}
      \begin{block}{2. Locker-Ransomware}
        \small
        \begin{itemize}
          \item Sperrt Systemzugriff
          \item Bildschirmsperre
          \item Keine Datenverschlüsselung
          \item Einfacher zu beheben
        \end{itemize}
      \end{block}
    \end{column}
  \end{columns}
  \vspace{0.3em}
  \begin{exampleblock}{Projekt-Einblick: Edu-Ransomware}
    \footnotesize
    Unsere Ransomware ist eine klassische \textbf{Crypto-Ransomware} mit \textbf{Double Extortion}.
  \end{exampleblock}
\end{frame}

% ============================================================
% SEKTION: CASE STUDY EINFÜHRUNG
% ============================================================
\section{Case Study: Edu-Ransomware}

\begin{frame}{Projektübersicht}
  \centering
  \includegraphics[width=0.85\textwidth,height=0.8\textheight,keepaspectratio]{images/01_gesamtarchitektur.png}
\end{frame}

% ============================================================
% SEKTION:AUfbau  
% ============================================================
\section{Technische Architektur}

\begin{frame}{Systemarchitektur \& Technologie-Stack}
\footnotesize
\centering
\textit{Modulare Client--Server-Architektur (Reverse TCP Shell)}

\begin{columns}[T,onlytextwidth]

% ===== AGENT =====
\begin{column}{0.49\textwidth}
\begin{block}{Agent \textnormal{(Client)} \hfill \textbf{Rust}}
\begin{itemize}
  \item \textbf{Deployment}: Statische Binary (Windows \texttt{.exe}, Linux \texttt{.deb})
  \item \textbf{Core-Komponenten}:
  \begin{itemize}
    \item \texttt{main.rs} – Initialisierung, Daemonisierung
    \item \texttt{evasion.rs} – Ressourcen- \& Zeitmanipulationsprüfung
    \item \texttt{crypto.rs} – AES-256-CTR, atomare Dateioperationen
    \item \texttt{network.rs} – Raw TCP, Retry-Logik, Protokollparser
    \item \texttt{extortion.rs} – UI-Manipulation (Wallpaper, Browser)
  \end{itemize}
\end{itemize}
\end{block}
\end{column}

% ===== SERVER + INFRA =====
\begin{column}{0.49\textwidth}

\begin{block}{C2-Server \textnormal{(Control)} \hfill \textbf{Python}}
\begin{itemize}
  \item Multithreaded TCP-Server
  \item Sitzungsverwaltung paralleler Clients
  \item Datenaufnahme (Base64-Streams $\rightarrow$ \texttt{loot/})
  \item Interaktive CLI
\end{itemize}
\end{block}

\vspace{0.4em}

\begin{alertblock}{Delivery Infrastructure \hfill \textbf{Bash / Python}}
\begin{itemize}
  \item User-Agent-basierte Payload-Auslieferung
  \item Automatisierter Build- \& Konfigurationsprozess
\end{itemize}
\end{alertblock}

\end{column}
\end{columns}
\end{frame}


% ============================================================
% SEKTION: PHASEN EINES ANGRIFFS
% ============================================================
\section{Die Phasen eines Angriffs}

\begin{frame}{Attack Kill Chain -- Übersicht}
  \textbf{Ablauf der Schadsoftware (Code-Logik):}
  
  \vspace{1em}
  
  \begin{enumerate}
    \item \textbf{Entry Point} (\texttt{main.rs})
    \begin{itemize}
        \item Initialer Start des Programms auf dem Zielsystem.
    \end{itemize}
    
    \item \textbf{Sandbox Check} (\texttt{evasion.rs})
    \begin{itemize}
        \item \textbf{Erfolg (OK):} Malware erkennt keine Analyse-Umgebung $\rightarrow$ Fortfahren.
        \item \textbf{Fehlschlag (FAIL):} Malware beendet sich sofort (\textbf{Exit}), um Entdeckung zu vermeiden.
    \end{itemize}

    \item \textbf{Persistence} (\texttt{persistence.rs})
    \begin{itemize}
        \item Einrichten eines \textbf{Autostarts}, damit die Malware Neustarts überdauert.
    \end{itemize}

    \item \textbf{Command \& Control} (\texttt{network.rs})
    \begin{itemize}
        \item Aufbau des \textbf{C2 Loops} zur Kommunikation mit dem Angreifer-Server.
    \end{itemize}

    \item \textbf{Verschlüsselung} (\texttt{crypto.rs})
    \begin{itemize}
        \item Lokale Dateiverschlüsselung mittels \textbf{AES-256}.
    \end{itemize}

    \item \textbf{Erpressung} (\texttt{extortion.rs})
    \begin{itemize}
        \item Anzeige der \textbf{Ransom Note} (Erpresserschreiben) für das Opfer.
    \end{itemize}
  \end{enumerate}
\end{frame}

\begin{frame}{Phase 1: Distribution \& Infection}
  \textbf{Theorie:}
  Phishing (E-Mail-Anhänge, Links), Drive-by-Downloads

  \vspace{0.5em}

  \begin{demobox}{Live Demo: Initial-Access-Szenario}
    \textbf{Szenario:}
    Mitarbeiter erhält eine E-Mail mit dem Anhang
    Rechnung\_Dez.pdf

    \vspace{0.5em}

    \begin{columns}
      \begin{column}{0.49\textwidth}
        \begin{block}{PDF}
          \begin{itemize}
            \item \textbf{Köder:}
                  PDF enthält ein unscharfes Bild einer Rechnung
            \item \textbf{Trick:}
                  Button ``Inhalt entschlüsseln''
                  (Social Engineering)
            \item \textbf{Ergebnis:}
                  Smishing-Server liefert Malware aus
          \end{itemize}
        \end{block}
      \end{column}

      \begin{column}{0.49\textwidth}
        \begin{block}{Website}
          \begin{itemize}
            \item \textbf{Köder:}
                  Unterschiedliche Websites fordern zum Download auf
            \item \textbf{Ergebnis:}
                  Server triggert den Download der Malware
          \end{itemize}
        \end{block}
      \end{column}
    \end{columns}
  \end{demobox}
\end{frame}

\begin{frame}{Phase 1: Drive-by-Download Ablauf}
  \centering
  \includegraphics[width=0.95\textwidth,height=0.82\textheight,keepaspectratio]{images/04_drive_by_download.png}
\end{frame}

\begin{frame}{Phase 1: PDF Phishing Technik}
  \centering
  \includegraphics[width=0.95\textwidth,height=0.82\textheight,keepaspectratio]{images/05_pdf_phishing.png}
\end{frame}

\begin{frame}{Phase 2: Execution \& Evasion}
  \textbf{Theorie:} Ransomware installiert sich, prüft auf Sandboxes, etabliert Persistenz.
  
  \vspace{0.5em}
  
  \begin{demobox}{Projekt-Einblick: Evasion Module}
    Der Agent prüft beim Start:
    \begin{itemize}
      \item Ist RAM < 3GB?
      \item Sind weniger als 2 CPU-Kerne verfügbar?
      \item Sind weniger als 60GB Festplattenspeicher verfügbar?
    \end{itemize}
    Falls ja: \alert{Sofortiger Abbruch} mit gefälschter Fehlermeldung.
  \end{demobox}
\end{frame}

\begin{frame}{Phase 2: Agent Architektur}
  \centering
  \includegraphics[width=0.95\textwidth,height=0.9\textheight,keepaspectratio]{images/02_agent_architecture.png}
\end{frame}

\begin{frame}{Phase 3: C2 \& Exfiltration}
  \textbf{Theorie:} Aufbau der Kommunikation, Nachladen von Befehlen, Datendiebstahl.
  
  \vspace{0.5em}
  
  \begin{demobox}{Live Demo: Attacker Control}
    Wir wechseln zum Angreifer-Terminal (C2):
    \begin{itemize}
      \item \code{[+] New Victim Connected: ID 1}
      \item Angreifer nutzt \code{shell}-Befehle zur Erkundung
      \item \textbf{Double Extortion:} \code{exfil secret.pdf} stiehlt Daten
    \end{itemize}
  \end{demobox}
\end{frame}

\begin{frame}{Phase 3: C2-Architektur (Reverse Shell)}
    \begin{columns}[c] % Zentriert ausgerichtet

        % Linke Spalte: Kompakter Text
        \begin{column}{0.45\textwidth}
            \begin{alertblock}{Das Prinzip}
                Verbindung von \textbf{Innen nach Außen} umgeht Firewall-Regeln.
            \end{alertblock}

            \vspace{1em}
            
            \textbf{Befehlssatz:}
            \begin{description}[font=\ttfamily\color{blue}]
                \item[shell] Führt Systembefehle aus
                \item[exfil] Stiehlt Dateien (Base64)
                \item[encrypt] Startet Verschlüsselung
                \item[decrypt] Startet Entschlüsselung 
            \end{description}
        \end{column}

        % Rechte Spalte: Vertikales Diagramm (besser für Spalten)
        \begin{column}{0.55\textwidth}
            \centering
            \resizebox{0.8\textwidth}{!}{
            \begin{tikzpicture}[node distance=1.5cm, font=\sffamily]
                % Komponenten untereinander statt nebeneinander
                \node[rectangle, draw, fill=red!10, minimum width=3cm] (c2) {C2 Server (Internet)};
                
                \node[rectangle, draw=red, dashed, fill=white, minimum width=4cm, minimum height=0.5cm, below=1cm of c2] (fw) {Firewall};
                
                \node[rectangle, draw, fill=green!10, minimum width=3cm, below=1cm of fw] (agent) {Opfer (Intranet)};

                % Pfeile
                % Grün: Raus
                \draw[->, ultra thick, green!60!black] (agent.east) to[out=0,in=0] node[right, font=\footnotesize]{1. Connect} (c2.east);
                
                % Rot: Rein (Blockiert)
                \draw[->, ultra thick, red] (c2.west) to[out=180,in=180] node[left, font=\footnotesize]{Blocked} (fw.west);
                
                % Blau: Antwort durch Tunnel
                \draw[->, thick, blue, dashed] (c2) -- node[fill=white, font=\tiny]{2. Response} (agent);

            \end{tikzpicture}
            }
        \end{column}

    \end{columns}
\end{frame}

\begin{frame}{Phase 4: Encryption (Impact)}
  \textbf{Theorie:} Starke Verschlüsselung, Löschung von Backups.
      
  \vspace{0.5em}
      
  \begin{demobox}{Live Demo: The Panic Mode}
    Angreifer sendet \code{encrypt}. Auswirkungen auf Opfer-PC:
    \begin{itemize}
      \item Dateien erhalten Endung \code{.locked}
      \item Browser öffnet Lösegeldforderung (Stress)
      \item Log-File zeigt Verschlüsselung in Echtzeit
    \end{itemize}
  \end{demobox}
\end{frame}

\begin{frame}{Phase 4: Verschlüsselungsprozess}
  \centering
  \includegraphics[width=0.95\textwidth,height=0.82\textheight,keepaspectratio]{images/05_encryption_02.png}
\end{frame}

\begin{frame}{Phase 5: Decryption (Recovery)}
  \textbf{Szenario:} Das Lösegeld wurde gezahlt (in der Simulation).
    
  \vspace{0.5em}
  
  \begin{itemize}
    \item Angreifer sendet Befehl \code{decrypt}
    \item Agent nutzt symmetrischen Key (AES-CTR)
    \item \code{.locked} Dateien verschwinden, Originale sind wieder da
  \end{itemize}
\end{frame}

% ============================================================
% SEKTION 5: GEGENMAßNAHMEN
% ============================================================
\section{Gegenma{\ss}nahmen}

\begin{frame}{Prävention \& Detektion}
  \textbf{Technische Ma{\ss}nahmen:}
  
  \begin{columns}[T]
    \begin{column}{0.48\textwidth}
      \begin{block}{Backups}
        \footnotesize 3-2-1-Regel: 3 Kopien, 2 Medien, 1 Offsite
      \end{block}
    \end{column}
    \begin{column}{0.48\textwidth}
      \begin{block}{EDR / Antivirus}
        \footnotesize Verhaltensanalyse in Echtzeit
      \end{block}
    \end{column}
  \end{columns}
  
  \vspace{0.5em}
  
  \begin{exampleblock}{Projekt-Reflektion: Warum Evasion?}
    \footnotesize
    Herkömmliche AV-Systeme scannen oft statisch. Unsere Ransomware umgeht dies durch:
    \begin{itemize}
      \item Sandbox-Checks (verhindert Cloud-Analyse)
      \item Nutzung von Rust (schwerer zu analysieren)
    \end{itemize}
  \end{exampleblock}
\end{frame}

\begin{frame}{Detektion und Reaktion}
  \begin{columns}[T]
    \begin{column}{0.48\textwidth}
      \textbf{Detektionsansätze:}
      \begin{itemize}
        \item \textbf{Statische Analyse}: Ohne Ausführung
        \item \textbf{Dynamische Analyse}: Sandbox
        \item \textbf{Machine Learning}: Unbekannte Varianten
        \item \textbf{Netzwerk}: Anomalie-Erkennung
      \end{itemize}
    \end{column}
    \begin{column}{0.48\textwidth}
      \textbf{Wichtige Merkmale:}
      \begin{itemize}
        \item API-Aufrufe und Systemverhalten
        \item Datei-/Verzeichnisaktivitäten
        \item Netzwerkverkehrsmuster
        \item Verschlüsselungsoperationen
      \end{itemize}
    \end{column}
  \end{columns}
\end{frame}

\begin{frame}{Incident Response}
  \begin{columns}[T]
    \begin{column}{0.48\textwidth}
      \textbf{Sofortma{\ss}nahmen bei Verdacht:}
      \begin{enumerate}
        \item \alert{Isolation} betroffener Systeme
        \item Identifikation des Ransomware-Stamms
        \item Bewertung der Schadenausbreitung
        \item Benachrichtigung relevanter Stellen
      \end{enumerate}
    \end{column}
    \begin{column}{0.48\textwidth}
      \textbf{Wiederherstellung:}
      \begin{itemize}
        \item Aus Backups (wenn möglich)
        \item Key-Escrow-Mechanismen
        \item Forensische Analyse
        \item Systemhärtung vor Neustart
      \end{itemize}
    \end{column}
  \end{columns}
  
  \vspace{0.5em}
  \begin{center}
    \alert{\textbf{Keine Lösegeldzahlung}} (keine Garantie, finanziert weitere Angriffe)
  \end{center}
\end{frame}

% ============================================================
% SEKTION 6: FAZIT
% ============================================================
\section{Fazit}

\begin{frame}{Zusammenfassung}
  \textbf{Erkenntnisse aus der Projektentwicklung:}
  \vspace{0.5em}
  
  \begin{columns}[T]
    \begin{column}{0.32\textwidth}
      \begin{block}{Komplexität}
        \footnotesize
        Als Anfänger von Null funktionsfähige Malware entwickelt -- erschreckend niedrige Einstiegshürde.
      \end{block}
    \end{column}
    \begin{column}{0.32\textwidth}
      \begin{block}{KI-Unterstützung}
        \footnotesize
        Entwicklung mit KI-Tools beschleunigt den Prozess erheblich.
      \end{block}
    \end{column}
    \begin{column}{0.32\textwidth}
      \begin{block}{AV-Evasion}
        \footnotesize
        \alert{Malware-Scanner schlägt nicht an} -- Agent bleibt unerkannt.
      \end{block}
    \end{column}
  \end{columns}
  
  \vspace{0.8em}
  
  \begin{warnbox}{Fazit}
    Unser Ziel war es herauszufinden, wie komplex die Implementierung funktionsfähiger Ransomware für Einsteiger ist. Das Ergebnis: \textbf{Mit modernen Tools und KI-Unterstützung ist die Hürde erschreckend niedrig.} Selbst gängige AV-Lösungen erkennen unseren Agent nicht.
  \end{warnbox}
\end{frame}

\begin{frame}{Zukünftige Herausforderungen}
  \begin{columns}[T]
    \begin{column}{0.48\textwidth}
      \textbf{Technologische Entwicklungen:}
      \begin{itemize}
        \item \textbf{KI in Ransomware}: Adaptives Verhalten
        \item \textbf{Quantencomputing}: Bedrohung für Verschlüsselung
        \item \textbf{IoT-Ransomware}: Neue Angriffsflächen
        \item \textbf{Cloud-native}: Angriffe auf Cloud-Infra
      \end{itemize}
    \end{column}
    \begin{column}{0.48\textwidth}
      \textbf{Forschungsbedarf:}
      \begin{itemize}
        \item Echtzeit-Schutz, Zero-Day-Erkennung
        \item Post-Quantum-Kryptographie
        \item Automatisierte Incident Response
        \item Internationale Strafverfolgung
      \end{itemize}
    \end{column}
  \end{columns}
\end{frame}

% ============================================================
% QUELLEN
% ============================================================
\begin{frame}{Quellen}
  \begin{thebibliography}{99}
    \setbeamertemplate{bibliography item}[text]
    \bibitem{hammann} IT-Security Vorlesung, Prof. Dr. Matthias Hamann, WS24/25.
    \bibitem{age} \textit{The Age of Ransomware: A Survey on the Evolution, Taxonomy, and Research Directions} (Razulla et al.)
    \bibitem{enisa} ENISA (2024): \textit{Threat Landscape Report 2024}
    \bibitem{project} \textit{Edu-Ransomware Repository} (GitHub, 2026) -- https://github.com/TimBLukas/rust-mw
    \bibitem{morato} Morato, D.; Berrueta, E.; Magaña, E.; Izal, M.: \textit{Ransomware early detection by the analysis of file sharing traffic}.
    \bibitem{oz} Oz, H.; Aris, A.; Levi, A.; Uluagac, A. S.: \textit{A Survey on Ransomware: Evolution, Taxonomy, and Defense Solutions}.
  \end{thebibliography}
\end{frame}

%==========================================Angriffskette=====================================
% 
% \begin{frame}{Die Angriffskette}
%   \centering
%   \begin{figure}
%     % Bild auf volle Breite vergrößert
%     \includegraphics[width=\textwidth]{images/}
%   \end{figure}
%   
%   \vspace{0.2em}
%   % Einheitlich: infobox UNTER dem Bild
%   \begin{infobox}{Infrastruktur-Übersicht}
%     Die Zustellung erfolgt dynamisch über einen Smart-Server (OS-Detection), während die Steuerung (C2) über verschleierte TCP-Tunnel realisiert wird.
%   \end{infobox}
% \end{frame}
% 
% %===========================================Netzwerk-Kommunikation===============================
% 
% \begin{frame}{Netzwerk-Kommunikation (Reverse Shell)}
%   \centering
%   \begin{figure}
%     % Bild auf volle Breite vergrößert
%     \includegraphics[width=\textwidth]{images/Network_Communication.png}
%   \end{figure}
%   
%   \vspace{0.2em}
%   % Einheitlich: infobox (statt demobox) UNTER dem Bild
%   \begin{infobox}{Kommunikationsfluss}
%     Der Rust-Agent baut einen verschlüsselten Callback zum C2-Server auf. Hierbei wird ein Pinggy-Tunnel genutzt, um die Firewall des Opfers zu umgehen.
%   \end{infobox}
% \end{frame}
% 
% %============================================Der Verschlüsselungsprozess (Atomic Encryption)=============
% 
% \begin{frame}{Der Verschlüsselungsprozess (Atomic Encryption)}
%   \begin{columns}[T]
%     % Linke Spalte: Das Bild
%     \begin{column}{0.55\textwidth}
%       \begin{figure}
%         \centering
%         \includegraphics[width=\textwidth]{images/Verschlusselungsprozess.png}
%       \end{figure}
%     \end{column}
%     
%     % Rechte Spalte: Die Textbox
%     \begin{column}{0.4\textwidth}
%       \vspace{1cm}
%       % Box-Farbe von Gelb (warnbox) auf Blau (infobox) geändert
%       \begin{infobox}{Technischer Ablauf}
%         \begin{itemize}
%           \item Algorithmus: AES-256-CTR
%           \item Symmetrischer Schlüssel
%           \item Erstellung der \code{.locked} Dateien
%           \item Löschung der Originale
%           \item Atomar = Ganz oder gar nicht
%         \end{itemize}
%       \end{infobox}
%     \end{column}
%   \end{columns}
% \end{frame}

\end{document}
