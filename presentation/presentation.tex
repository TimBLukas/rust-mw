\documentclass[aspectratio=169, 10pt]{beamer}

% --- Theme & Design ---
\usetheme[progressbar=frametitle, block=fill]{metropolis}

% --- Farben (Modernes Security-Blau Schema) ---
\definecolor{SecBlue}{RGB}{10, 30, 60}       % Sehr dunkles Blau (Background/Text)
\definecolor{TechBlue}{RGB}{0, 85, 150}      % Mittleres Blau (Elemente)
\definecolor{CyberCyan}{RGB}{0, 180, 216}    % Helles Cyan (Highlights)
\definecolor{LightGrey}{RGB}{248, 249, 250}  % Fast Weiß
\definecolor{AlertRed}{RGB}{200, 40, 40}     % Für Bedrohungen

% Farb-Anpassungen des Themes
\setbeamercolor{frametitle}{bg=white, fg=TechBlue}
\setbeamercolor{title separator}{fg=CyberCyan}
\setbeamercolor{progress bar}{fg=CyberCyan, bg=LightGrey}
\setbeamercolor{alerted text}{fg=AlertRed}
\setbeamercolor{normal text}{fg=SecBlue}
\setbeamercolor{block title}{bg=TechBlue, fg=white}
\setbeamercolor{block body}{bg=LightGrey, fg=SecBlue}

% --- Pakete ---
\usepackage[utf8]{inputenc}
\usepackage[ngerman]{babel}
\usepackage{helvet} 
\usepackage{listings}
\usepackage{tikz}
\usetikzlibrary{shapes, arrows, positioning, shadows, calc, fit, shapes.geometric, patterns, matrix}
\usetikzlibrary{shapes, shapes.symbols, arrows, positioning, shadows, calc, fit, patterns, matrix}
\usepackage{booktabs}
\usepackage{pifont} % Für Checkmarks

% --- Code Listing Style ---
\lstset{
    basicstyle=\ttfamily\scriptsize\color{SecBlue},
    keywordstyle=\color{TechBlue}\bfseries,
    commentstyle=\color{gray!80},
    stringstyle=\color{teal},
    numberstyle=\tiny\color{gray},
    breaklines=true,
    frame=l,
    framesep=5pt,
    framerule=2pt,
    rulecolor=\color{CyberCyan},
    backgroundcolor=\color{LightGrey},
    showstringspaces=false,
    language=bash
}

% --- Meta Data ---
\title{Ransomware: Analyse \& Simulation}
\subtitle{Bedrohungslage, Kill-Chain und technische Implementierung (Rust/Python)}
\author{\textbf{Demo Präsentation}}
\date{\today}
\titlegraphic{\hfill\footnotesize\ttfamily\color{gray}Security Research v1.0}

\begin{document}
\shorthandoff{"} % Wichtig für Babel/TikZ Kompatibilität

% 1. Title Page
\maketitle

% ==============================================================================
% TEIL 1: THEORIE & BEDROHUNGSLAGE (ERWEITERT)
% ==============================================================================

\section{Teil I: Das Ökosystem Ransomware}

% --- Folie 1: Definition & Evolution ---
\begin{frame}{Was ist Ransomware?}
    \begin{columns}[T,onlytextwidth]
        \column{0.6\textwidth}
            \textbf{Definition:}
            Malware, die den Zugriff auf Daten oder Systeme durch Verschlüsselung verhindert und für die Freigabe ein Lösegeld fordert.
            
            \vspace{1em}
            \textbf{Die Evolution der Erpressung:}
            \begin{itemize}
                \item[\textbf{Gen 1:}] \textbf{Single Extortion:} Nur Verschlüsselung.
                \item[\textbf{Gen 2:}] \textbf{Double Extortion:} + Diebstahl sensibler Daten (Drohung mit Leak-Seiten).
                \item[\textbf{Gen 3:}] \textbf{Triple Extortion:} + DDoS-Attacken oder Belästigung von Kunden/Partnern des Opfers.
            \end{itemize}

        \column{0.4\textwidth}
            \centering
            % TikZ: Die Eskalationsstufen
            \begin{tikzpicture}[node distance=0.5cm]
                \node[draw=TechBlue, fill=LightGrey, rounded corners, minimum width=2.5cm] (L1) {Encryption};
                \node[draw=TechBlue, fill=CyberCyan!30, rounded corners, minimum width=2.5cm, below=of L1] (L2) {Data Leak};
                \node[draw=AlertRed, fill=AlertRed!10, rounded corners, minimum width=2.5cm, below=of L2] (L3) {DDoS / Harassment};
                
                \draw[->, thick, color=gray] (L1) -- (L2);
                \draw[->, thick, color=gray] (L2) -- (L3);
            \end{tikzpicture}
    \end{columns}
\end{frame}

% --- Folie 2: RaaS Geschäftsmodell ---
\begin{frame}{Das Geschäftsmodell: Ransomware-as-a-Service (RaaS)}
    Ransomware ist heute eine hochprofessionelle Industrie mit Arbeitsteilung.
    
    \vspace{1em}
    \centering
    \begin{tikzpicture}[
        node distance=1.5cm,
        actor/.style={rectangle, draw=SecBlue, thick, fill=white, rounded corners, minimum height=1.2cm, minimum width=3cm, align=center, drop shadow},
        arrow/.style={->, >=stealth, thick, color=TechBlue}
    ]
        % Actors
        \node[actor] (dev) {\textbf{Core Developers}\\(Operator)};
        \node[actor, right=3cm of dev] (aff) {\textbf{Affiliates}\\(Angreifer)};
        \node[actor, below=1.5cm of aff, draw=AlertRed] (victim) {\textbf{Opfer}\\(Unternehmen)};
        
        % Relations
        \draw[arrow] ([yshift=5pt]dev.east) -- node[above, font=\tiny] {Stellt Malware \& C2} ([yshift=5pt]aff.west);
        \draw[arrow] ([yshift=-5pt]aff.west) -- node[below, font=\tiny] {30\% Provision} ([yshift=-5pt]dev.east);
        
        \draw[arrow, color=AlertRed] (aff) -- node[right, font=\tiny] {Infektion} (victim);
        \draw[arrow, dashed] (victim) -- node[left, font=\tiny, align=right] {Lösegeld\\(Krypto)} ($(dev.south)!0.5!(aff.south)$);
        
    \end{tikzpicture}
    
    \vspace{1em}
    \footnotesize
    \textbf{Vorteil für Kriminelle:} Die Entwickler müssen sich nicht die Hände schmutzig machen (Einbruch), die Affiliates müssen nicht programmieren können.
\end{frame}

% --- Folie 3: Initial Access ---
\begin{frame}{Wie kommen sie rein? (Initial Access Vectors)}
    Die häufigsten Einfallstore für Ransomware-Affiliates:
    
    \vspace{1.5em}
    \centering
    \begin{tikzpicture}
        % Achsen
        \draw[gray!50] (0,0) -- (9,0);
        \draw[gray!50] (0,0) -- (0,3.5);
        
        % Balken 1: Phishing
        \draw[fill=TechBlue] (0, 2.8) rectangle (6.5, 3.3);
        \node[right, font=\footnotesize] at (6.5, 3.05) {Phishing / Social Eng. ($\sim$40\%)};
        
        % Balken 2: Vulnerabilities
        \draw[fill=CyberCyan] (0, 1.8) rectangle (4.5, 2.3);
        \node[right, font=\footnotesize] at (4.5, 2.05) {Schwachstellen (VPN/Citrix) ($\sim$25\%)};
        
        % Balken 3: RDP
        \draw[fill=SecBlue] (0, 0.8) rectangle (3.5, 1.3);
        \node[right, font=\footnotesize] at (3.5, 1.05) {Exposed RDP / Credential Stuffing ($\sim$20\%)};
        
        % Balken 4: Supply Chain
        \draw[fill=gray] (0, -0.2) rectangle (1.5, 0.3);
        \node[right, font=\footnotesize] at (1.5, 0.05) {Supply Chain / Sonstiges ($\sim$15\%)};
        
    \end{tikzpicture}
    
    \begin{tikzpicture}[remember picture, overlay]
        \node[anchor=south east, xshift=-10pt, yshift=5pt, font=\tiny\color{gray}] at (current page.south east) {Datenbasis: Aggregierte Reports 2023/2024 (Mandiant, Sophos)};
    \end{tikzpicture}
\end{frame}

% --- Folie 4: Technische Kill Chain ---
\begin{frame}{Die technische "Kill Chain"}
    Vom ersten Zugriff bis zur Verschlüsselung vergehen oft Wochen ("Dwell Time").
    
    \vspace{1.5em}
    \centering
    \begin{tikzpicture}[
        node distance=0.4cm,
        phase/.style={rectangle, draw=TechBlue, thick, fill=white, rounded corners, minimum width=2.6cm, minimum height=1.4cm, align=center, font=\scriptsize, drop shadow},
        arrow/.style={->, >=stealth, thick, color=CyberCyan, line width=1.5pt}
    ]
        % Phasen
        \node[phase] (access) {\textbf{1. Access}\\\tiny Phishing, RDP\\Exploits};
        \node[phase, right=of access] (persist) {\textbf{2. Persistence}\\\& C2\\\tiny Beaconing};
        \node[phase, right=of persist] (lat) {\textbf{3. Recon}\\\& Lateral Mov.\\\tiny AD-Übernahme};
        \node[phase, right=of lat, draw=AlertRed, fill=AlertRed!5] (impact) {\textbf{4. Impact}\\\tiny Data Exfil \&\\Encryption};
        
        % Pfeile
        \draw[arrow] (access) -- (persist);
        \draw[arrow] (persist) -- (lat);
        \draw[arrow] (lat) -- (impact);
        
    \end{tikzpicture}
    
    \vspace{1.5em}
    \begin{itemize}
        \item \textbf{Schritt 2 \& 3} sind entscheidend für die Verteidigung. Sobald Schritt 4 beginnt, ist es meist zu spät.
        \item Unsere Simulation überspringt \textbf{Schritt 3 (Recon \& lateral Movement)}.
    \end{itemize}
\end{frame}

% --- Folie 5: Krypto-Hintergrund ---
\begin{frame}{Exkurs: Warum wir den Key nicht einfach "finden"}
    Professionelle Ransomware nutzt \textbf{hybride Verschlüsselung}, um Geschwindigkeit mit Sicherheit zu kombinieren.
    
    \vspace{1em}
    \centering
    \begin{tikzpicture}[
        node distance=1cm,
        box/.style={rectangle, draw=gray, rounded corners, font=\tiny, align=center, minimum width=1.8cm, minimum height=1cm},
        arrow/.style={->, >=stealth, color=TechBlue}
    ]
        % Symmetrisch
        \node[box, draw=TechBlue] (file) {Datei};
        \node[box, right=of file] (aes) {\textbf{AES Key}\\(Symmetrisch)};
        \node[box, draw=AlertRed, right=of aes] (encfile) {Encrypted\\File};
        
        % Asymmetrisch
        \node[box, above=of aes] (pub) {\textbf{RSA Public Key}\\(Im Code)};
        \node[box, draw=AlertRed, right=of pub] (enckey) {Encrypted\\AES Key};
        
        % Pfeile
        \draw[arrow] (file) -- (aes);
        \draw[arrow] (aes) -- (encfile);
        \draw[arrow] (aes) -- (pub);
        \draw[arrow] (pub) -- (enckey);
        
        \node[right=0.2cm of encfile, font=\tiny, color=gray, align=left] {$\leftarrow$ Schnell,\\aber Key liegt im RAM};
        \node[right=0.2cm of enckey, font=\tiny, color=gray, align=left] {$\leftarrow$ Key ist sicher\\weggeschlossen};

    \end{tikzpicture}
    
    \vspace{1em}
    \raggedright
    \footnotesize
    \textbf{Das Problem:} Der AES-Key (zum Entschlüsseln nötig) wird mit dem Public Key des Angreifers verschlüsselt und aus dem Speicher gelöscht. Nur der Angreifer hat den Private Key zum Wiederherstellen.
    \newline
    \textit{Hinweis: In unserer Demo vereinfachen wir dies und speichern den Key lokal (`rescue.key`).}
\end{frame}

% --- Folie 6: MITRE ATT&CK ---
\begin{frame}{MITRE ATT\&CK Taktiken}
    \centering
    \footnotesize
    \begin{tabular}{p{3.5cm} p{3.5cm} p{3.5cm} p{3.5cm}}
        \toprule
        \textcolor{TechBlue}{\textbf{Initial Access}} & \textcolor{TechBlue}{\textbf{Persistence}} & \textcolor{TechBlue}{\textbf{Command \& Control}} & \textcolor{AlertRed}{\textbf{Impact}} \\
        \midrule
        $\bullet$ Phishing (T1566) & $\bullet$ Reg Run Keys & $\bullet$ Web Protocols & $\bullet$ Data Encrypted \\
        $\bullet$ Exploit Public App & $\bullet$ Create Account & $\bullet$ Non-Standard Port & \quad for Impact \\
        $\bullet$ Valid Accounts & $\bullet$ Scheduled Task & $\bullet$ Encrypted Channel & $\bullet$ Service Stop \\
        \bottomrule
    \end{tabular}

    \vspace{2em}
    \raggedright
    \textbf{Mapping zur Demo:}
    Unsere Rust-Malware nutzt \textit{Registry Run Keys / Systemd} für Persistence und \textit{Non-Standard Ports} (TCP Socket) für C2.
\end{frame}

% --- Folie 5: C2 Context ---
\begin{frame}{Command \& Control (C2) Architektur}
    Die "Lebensader" des Angriffs: C2 steuert Payloads, empfängt Keys und exfiltrierte Daten.
    
    \vspace{1em}
    \centering
    \begin{tikzpicture}[node distance=1.5cm]
        % Nodes
        \node[draw=SecBlue, fill=LightGrey, rounded corners, align=center] (corp) {
            \textbf{Corporate Network}\\
            \scriptsize Client A \quad Client B
        };
        
        \node[draw=AlertRed, thick, fill=white, right=2cm of corp, align=center, rounded corners] (infected) {Infected\\Host};
        
        % KORREKTUR: Cloud als Node-Shape, nicht als Befehl
        \node[cloud, cloud puffs=10, draw=gray, aspect=2, right=2cm of infected, align=center] (internet) {Internet};
        
        \node[draw=TechBlue, fill=SecBlue, text=white, rounded corners, right=0.5cm of internet] (c2) {C2 Server};
        
        % Connections
        \draw[thick] (corp) -- (infected);
        \draw[<->, dashed, thick, color=AlertRed] (infected) -- node[above, font=\tiny] {HTTPS Beaconing} node[below, font=\tiny] {DNS Tunneling} (internet);
        \draw[<->, thick] (internet) -- (c2);
        
    \end{tikzpicture}
    
    \vspace{1em}
    \raggedright
    \small
    Nutzung unauffälliger Kanäle (Port 443, DNS), um in der Masse des Traffics unterzugehen.

    \begin{tikzpicture}[remember picture, overlay]
        \node[anchor=south east, xshift=-10pt, yshift=5pt, font=\tiny\color{gray}] at (current page.south east) {Quellen: Halcyon – C2 in Ransomware; Packetlabs Deep Dive};
    \end{tikzpicture}
\end{frame}


% --- Folie 6: Defense ---
\begin{frame}{Erkennung \& Abwehr (Defense in Depth)}
    
    \begin{columns}
        \column{0.6\textwidth}
            \textbf{Technische Maßnahmen:}
            \begin{itemize}
                \item \textbf{C2-Detection:} Blockieren von bekannten Malicious IPs/Domains \& Analyse von Beaconing-Mustern.
                \item \textbf{EDR (Endpoint Detection):} Überwachung auf Prozess-Injections und Massen-Dateiänderungen.
            \end{itemize}
            
            \vspace{0.5em}
            \textbf{Organisatorisch:}
            \begin{itemize}
                \item Offline-Backups (Immutable).
                \item Netzwerksegmentierung.
            \end{itemize}

        \column{0.4\textwidth}
            \centering
            % TikZ Layers
            \begin{tikzpicture}
                \node[draw=TechBlue, fill=white, minimum width=3cm, rounded corners] (L1) {Awareness};
                \node[draw=TechBlue, fill=LightGrey, minimum width=3cm, rounded corners, above=0.2cm of L1] (L2) {Firewall / IDS};
                \node[draw=TechBlue, fill=CyberCyan!30, minimum width=3cm, rounded corners, above=0.2cm of L2] (L3) {EDR / AV};
                \node[draw=SecBlue, fill=TechBlue, text=white, minimum width=3cm, rounded corners, above=0.2cm of L3] (L4) {Backup};
                
                \node[right=0.2cm of L4, font=\tiny, color=gray] {Letzte Linie};
            \end{tikzpicture}
    \end{columns}

    \begin{tikzpicture}[remember picture, overlay]
        \node[anchor=south east, xshift=-10pt, yshift=5pt, font=\tiny\color{gray}] at (current page.south east) {Quellen: BSI; Darktrace – Early Signs of Ransomware};
    \end{tikzpicture}
\end{frame}


% ==============================================================================
% TEIL 2: TECHNISCHE SIMULATION (Praxis)
% ==============================================================================

\section{Teil II: Technische Simulation}

\begin{frame}{Architektur der Simulation}
    \centering
    \begin{tikzpicture}[
        node distance=2cm,
        box/.style={rectangle, draw=none, fill=LightGrey, rounded corners, minimum height=1.5cm, minimum width=3.5cm, align=center, drop shadow},
        label/.style={font=\tiny, color=gray},
        arrow/.style={->, >=stealth, thick, color=TechBlue}
    ]
        % Nodes
        \node[box] (victim) {\textbf{\textcolor{CyberCyan}{RUST AGENT}}\\(Victim Client)};
        \node[box, right=4cm of victim] (c2) {\textbf{\textcolor{TechBlue}{PYTHON C2}}\\(Attacker Server)};
        
        % Firewall Wall Visualisierung
        \draw[line width=2pt, dashed, color=gray!50] (4, -1.8) -- (4, 1.8);
        \node[fill=white, text=gray, font=\tiny, inner sep=2pt] at (4,0) {Firewall / NAT};
        
        % Arrows
        \draw[arrow] ([yshift=8pt]victim.east) -- node[above, font=\tiny, align=center] {1. Reverse TCP Connect\\(Port 4444)} ([yshift=8pt]c2.west);
        \draw[arrow, dashed] ([yshift=-8pt]c2.west) -- node[below, font=\tiny, align=center] {2. Commands\\(encrypt/decrypt)} ([yshift=-8pt]victim.east);
        
    \end{tikzpicture}

    \vspace{1.5em}
    \begin{itemize}
        \item \textbf{Reverse Shell Prinzip:} Der Agent baut die Verbindung auf (Outbound), um Inbound-Firewall-Regeln zu umgehen.
        \item \textbf{Technologie:} Rust (Client) für Performance und Sicherheit, Python (Server) für einfache Handhabung.
    \end{itemize}
\end{frame}

\begin{frame}{Der Agent: Warum Rust?}
    \begin{columns}[T,onlytextwidth]
        \column{0.5\textwidth}
            \begin{block}{Technische Vorteile}
                \begin{itemize}
                    \item \textbf{Memory Safety:} Keine Buffer Overflows (weniger Abstürze beim Opfer).
                    \item \textbf{Zero Runtime:} Keine Installation von Python/Java nötig (Single Binary).
                    \item \textbf{Cross-Platform:} Ein Code für Linux, Windows \& macOS.
                \end{itemize}
            \end{block}
        
        \column{0.45\textwidth}
            \begin{block}{Evasion (Erkennung)}
                \begin{itemize}
                    \item Komplexer Maschinencode erschwert Reverse Engineering.
                    \item Geringere Erkennungsrate bei klassischen AV-Signaturen im Vergleich zu C/C++.
                \end{itemize}
            \end{block}
    \end{columns}
\end{frame}

% Cryptography Flow
\begin{frame}{Kryptographie: Der "Atomic Lock"}
    Ein kritischer Fehler bei Malware ist Datenverlust durch Abstürze. Unser Agent nutzt ein atomares Verfahren:

    \vspace{1.5em}
    \centering
    \begin{tikzpicture}[
        node distance=0.5cm,
        proc/.style={rectangle, draw=TechBlue, thick, fill=white, rounded corners, minimum width=2.2cm, minimum height=1.2cm, align=center, font=\scriptsize},
        arrow/.style={->, >=stealth, thick, color=CyberCyan}
    ]
        \node[proc] (read) {1. Stream Read\\(4KB Chunks)};
        \node[proc, right=of read] (enc) {2. AES-256-CTR\\(XOR Stream)};
        \node[proc, right=of enc] (write) {3. Write Temp\\\texttt{.enc\_temp}};
        \node[proc, right=of write, fill=TechBlue, text=white] (rename) {4. Atomic Rename\\\texttt{.locked}};

        \draw[arrow] (read) -- (enc);
        \draw[arrow] (enc) -- (write);
        \draw[arrow] (write) -- (rename);
        
        % Loop annotation
        \draw[->, gray, dashed] ($(write.south) + (-0.2, 0)$) to[out=240,in=300] node[below, font=\tiny] {Loop bis EOF} ($(read.south) + (0.2, 0)$);
    \end{tikzpicture}
    
    \vspace{1.5em}
    \raggedright
    \footnotesize
    \textbf{AES-256-CTR:} Der Counter-Mode ermöglicht schnelles Verschlüsseln ohne "Padding" (Dateigröße bleibt identisch).
\end{frame}

\begin{frame}[fragile]{Persistenz: Das Überleben des Neustarts}
    Malware muss sich im System verankern. Der Rust-Code unterscheidet zur Kompilierzeit das OS:
    
    \begin{columns}
        \column{0.5\textwidth}
            \textbf{Windows (Registry)}
            \begin{lstlisting}
Command::new("reg")
    .args([
        "add", 
        "HKCU\\...\\Run",
        "/v", "Updater",
        "/d", exe_path
    ])
            \end{lstlisting}
            
        \column{0.5\textwidth}
            \textbf{Linux (Systemd)}
            \begin{lstlisting}
[Unit]
Description=SystemSvc

[Service]
ExecStart=/path/to/agent
Restart=always
            \end{lstlisting}
    \end{columns}
\end{frame}

% --- Demo Workflow ---
\begin{frame}{Ablauf der Demonstration}
    Wie die Komponenten im Test zusammenspielen:
    
    \vspace{1em}
    
    \begin{enumerate}
        \item \textbf{Setup:} Starten des Python C2 Servers (Listener auf Port 4444).
        \item \textbf{Infektion:} Ausführen der \texttt{rust-mw} Binary auf dem Zielsystem.
        \item \textbf{Handshake:} Client meldet sich beim Server ("New Session").
        \item \textbf{Angriff:} 
            \begin{itemize}
                \item Server sendet Befehl: \texttt{encrypt}
                \item Client generiert Key $\rightarrow$ Verschlüsselt Dateien $\rightarrow$ Zeigt Lösegeldforderung.
            \end{itemize}
        \item \textbf{Lösung:} Server sendet Befehl \texttt{decrypt} (simulierte Zahlung).
    \end{enumerate}
\end{frame}

% Conclusion
\begin{frame}[standout]
    \Huge Fazit
    
    \normalsize
    \vspace{2em}
    \begin{itemize}
        \item[\checkmark] \textbf{Bedrohung:} Ransomware professionalisiert sich (RaaS, C2).
        \item[\checkmark] \textbf{Technik:} Rust + Sockets ermöglichen potente Malware.
        \item[\checkmark] \textbf{Schutz:} Defense-in-Depth ist alternativlos.
    \end{itemize}
\end{frame}

\end{document}
