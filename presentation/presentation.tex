\documentclass[aspectratio=169]{beamer}

% Theme und Packages
\usetheme{metropolis}
\usepackage[utf8]{inputenc}
\usepackage[T1]{fontenc}
\usepackage[ngerman]{babel}
\usepackage{graphicx}
\usepackage{booktabs}
\usepackage{xcolor}
\usepackage{tikz}
\usepackage{pgfplots}
\pgfplotsset{compat=1.18}
\usepackage[edges]{forest}
\usetikzlibrary{shapes,arrows,positioning,fit,backgrounds}
\usepackage{tcolorbox}
\newtcolorbox{demobox}[1]{
  colback=white,
  colframe=darkaccent,         % Rahmenfarbe
  coltext=darkaccent,               % Textfarbe: Weiß
  title=\textbf{#1},           % Titel fett
  coltitle=orange,             % Titelfarbe: Orange
  fonttitle=\sffamily,         % Titel-Schrift: Normal (Sans-Serif)
  fontupper=\sffamily\small,   % Text-Schrift: Normal (Sans-Serif) & klein
  arc=0pt,                     % Eckige Kanten
  boxsep=0pt,
  toptitle=2pt,
  bottomtitle=2pt,
  outer arc=0pt,
  boxrule=1pt,                 % Kein Rand
  left=10pt, right=10pt, top=2pt, bottom=8pt % Etwas mehr Innenabstand für Text
}

% Farbschema
\definecolor{darkaccent}{RGB}{35,55,75}
\definecolor{chartcolor1}{RGB}{231,76,60}
\definecolor{chartcolor2}{RGB}{52,152,219}
\definecolor{chartcolor3}{RGB}{46,204,113}
\definecolor{chartcolor4}{RGB}{241,196,15}
\definecolor{chartcolor5}{RGB}{155,89,182}
\definecolor{chartcolor6}{RGB}{230,126,34}

% Custom Colors für Projekt-Verweise
\definecolor{rustcolor}{RGB}{183,65,14}
\setbeamercolor{block title example}{fg=white, bg=rustcolor}
\setbeamercolor{block body example}{bg=rustcolor!10}

\setbeamercolor{frametitle}{bg=darkaccent}
\setbeamercolor{progress bar}{fg=orange,bg=gray}

% Titelseite Informationen
\title{Ransomware}
\subtitle{Anatomie, Netzwerkerkennung und Zero-Loss Recovery}
\author{Ihr Name}
\date{\today}
\institute{Ihre Institution}

\begin{document}

% Titelfolie
\begin{frame}
  \titlepage
\end{frame}

% Agenda
\begin{frame}{Agenda}
  \tableofcontents
\end{frame}

% Sektion 1: Entwicklung & Geschichte
\section{Entwicklung \& Geschichte}

\begin{frame}{Historischer Abriss}
  \begin{columns}
    \begin{column}{0.5\textwidth}
      \textbf{Meilensteine:}
      \begin{itemize}
        \item \textbf{1989}: AIDS Trojan - erster dokumentierter Ransomware-Angriff
        \item \textbf{2005}: Wiederaufleben durch Internet und Kryptowährungen
        \item \textbf{2017}: WannaCry-Ausbruch mit globaler Aufmerksamkeit
      \end{itemize}
    \end{column}
    \begin{column}{0.5\textwidth}
      \begin{figure}
        % Placeholder für Bild
        \centering
        \begin{figure}
          \includegraphics[width=\textwidth]{images/historical_timeline.png}
          \caption{Entwicklung von Ransomware über die Jahre}
        \end{figure}
        \caption{Entwicklung von Ransomware über die Jahre}
      \end{figure}
    \end{column}
  \end{columns}
\end{frame}

% Sektion 2: Aktuelle Bedrohungslage
\section{Aktuelle Bedrohungslage}

\begin{frame}{Finanzielle Auswirkungen}
  \begin{columns}
    \begin{column}{0.45\textwidth}
      \textbf{Geschätzte Weltweite Schäden}
      \begin{itemize}
        \item 2017: 5 Mrd. USD
        \item 2024: 42 Mrd. USD
      \end{itemize}
      \vspace{0.5em}
    \end{column}
    \begin{column}{0.55\textwidth}
      \begin{tikzpicture}
        \begin{axis}[
          width=\textwidth,
          height=0.7\textheight,
          xlabel={Jahr},
          ylabel={Schaden (Mrd. USD)},
          ymin=0, ymax=45,
          xtick={2015,2017,2019,2021,2024},
          ytick={0,10,20,30,40},
          grid=major,
          legend pos=north west,
          ylabel near ticks,
          xlabel near ticks,
          x tick label style={/pgf/number format/1000 sep={}}
        ]
        \addplot[
          color=chartcolor1,
          mark=*,
          line width=2pt
        ] coordinates {
          (2015,0.325)
          (2017,5)
          (2019,11.5)
          (2021,20)
          (2024,42)
        };
        \legend{Weltweite Schäden}
        \end{axis}
      \end{tikzpicture}
    \end{column}
  \end{columns}
\end{frame}

\begin{frame}{ENISA Threat Landscape 2024: Prime Threats}
  \begin{columns}
    \begin{column}{0.4\textwidth}
      \textbf{Top-Bedrohungen EU:}
      \begin{enumerate}
        \item DDoS/RDoS (46,31\%)
        \item \alert{Ransomware (27,33\%)}
        \item Data Breaches (15,87\%)
        \item Social Engineering (5,37\%)
        \item Malware (2,45\%)
        \item Zero-Day (0,11\%)
      \end{enumerate}
      \vspace{0.5em}
      Ransomware: Zweitgrößte Bedrohung
    \end{column}
    \begin{column}{0.6\textwidth}
      \begin{tikzpicture}
        \begin{axis}[
          width=0.9\textwidth,
          height=0.75\textheight,
          xbar,
          xlabel={Vorfälle (in Tausend)},
          symbolic y coords={Zero-Day, Malware, Social Eng., Data, Ransomware, DDoS},
          ytick=data,
          nodes near coords,
          nodes near coords align={horizontal},
          xmin=0, xmax=2.8,
          bar width=0.5cm,
          ylabel near ticks,
          xlabel near ticks
        ]
        \addplot[fill=chartcolor1] coordinates {
          (0.01,Zero-Day)
          (0.13,Malware)
          (0.29,Social Eng.)
          (0.84,Data)
          (1.45,Ransomware)
          (2.46,DDoS)
        };
        \end{axis}
      \end{tikzpicture}
    \end{column}
  \end{columns}
\end{frame}

% Sektion 3: Ziele & Motivation
\section{Ziele \& Motivation}

\begin{frame}{Am häufigsten betroffene Sektoren}
  \begin{columns}
    \begin{column}{0.4\textwidth}
      \textbf{Top 5 Sektoren:}
      \begin{enumerate}
        \item Manufacturing (21,78\%)
        \item Retail (11,06\%)
        \item ICT Services (9,18\%)
        \item Business Services (7,14\%)
        \item Banking/Finance (6,94\%)
      \end{enumerate}
      \vspace{0.5em}
      \small{Industrie und Fertigung am stärksten betroffen}
    \end{column}
    \begin{column}{0.6\textwidth}
      \begin{tikzpicture}
        \begin{axis}[
          width=0.9\textwidth,
          height=0.9\textheight,
          xbar,
          xlabel={Anteil (\%)},
          symbolic y coords={Water, Energy, Public Admin, Transport, Education, Banking, Business, ICT, Other, Retail, Manufacturing},
          ytick=data,
          nodes near coords,
          nodes near coords align={horizontal},
          xmin=0, xmax=24,
          bar width=0.3cm,
          ylabel near ticks,
          xlabel near ticks,
          every node near coord/.append style={font=\scriptsize}
        ]
        \addplot[fill=chartcolor2] coordinates {
          (0.37,Water)
          (2.78,Energy)
          (3.58,Public Admin)
          (3.97,Transport)
          (4.92,Education)
          (6.94,Banking)
          (7.14,Business)
          (9.18,ICT)
          (10.52,Other)
          (11.06,Retail)
          (21.78,Manufacturing)
        };
        \end{axis}
      \end{tikzpicture}
    \end{column}
  \end{columns}
\end{frame}

\begin{frame}{Motivation der Angreifer}
  \textbf{Warum diese Ziele?}
  \begin{itemize}
    \item \textbf{Finanzielle Motive}: Zahlungsbereitschaft bei kritischen Systemen
    \item \textbf{Kritische Infrastrukturen}: Hoher Druck durch Ausfallkosten
    \item \textbf{Sensible Daten}: Erpressungspotenzial durch Datenleaks
    \item \textbf{Geopolitische Faktoren}: Destabilisierung und Spionage
  \end{itemize}
  
  \vspace{0.5em}
  
  \alert{Zunehmende Professionalisierung und Organisation}
\end{frame}

% Sektion 4: Arten von Ransomware
\section{Arten von Ransomware}

\begin{frame}{Ransomware Taxonomy}
  % Wir definieren einen gemeinsamen Stil für beide Bäume
  \tikzset{
    mybox/.style={
      font=\scriptsize\sffamily\bfseries,
      draw=gray, line width=0.8pt, inner sep=3pt, rounded corners=2pt, align=center
    },
    greenbox/.style={mybox, fill=green!30!gray!40, text width=2.2cm, minimum height=1cm},
    bluebox/.style={mybox, rectangle, rounded corners=0, fill=cyan!20, text width=2.2cm, font=\scriptsize\sffamily}
  }

  % Zeile 1: Die ersten 3 Kategorien
  \centering
  \resizebox{0.95\textwidth}{!}{%
    \begin{forest}
      for tree={edge={draw, gray}, s sep=0.3cm, l sep=0.5cm}
      [, phantom, forked edges % Unsichtbare Wurzel
        [Types, greenbox, for children={folder, grow'=0} [Crypto, bluebox][Locker, bluebox][Scareware, bluebox]]
        [Targets, greenbox, for children={folder, grow'=0} [Platforms, bluebox][Industries, bluebox]]
        [Infection vectors, greenbox, for children={folder, grow'=0} [Windows RDP, bluebox][Phishing, bluebox][Software vuln., bluebox]]
      ]
    \end{forest}
  }

  \vspace{0.5cm} % Abstand zwischen den Zeilen

  % Zeile 2: Die restlichen 3 Kategorien
  \resizebox{0.95\textwidth}{!}{%
    \begin{forest}
      for tree={edge={draw, gray}, s sep=0.3cm, l sep=0.5cm}
      [, phantom, forked edges
        [Comm. with C\&C, greenbox, for children={folder, grow'=0} [Hard-coded IP, bluebox][DGA, bluebox]]
        [Access to enc. key, greenbox, for children={folder, grow'=0} [Hard-coded binary, bluebox][Via C\&C, bluebox]]
        [Malicious action, greenbox, for children={folder, grow'=0} [Encryption, bluebox][Locking, bluebox]]
      ]
    \end{forest}
  }
  
  \begin{exampleblock}{Projekt-Einblick: Edu-Ransomware}
    \vspace{0.2cm}
    Unsere entwickelte Ransomware ist eine klassische \textbf{Crypto-Ransomware} mit Elementen der \textbf{Double Extortion} (Datenexfiltration vor Verschlüsselung).
  \end{exampleblock}
\end{frame}

\begin{frame}{Grundlegende Typen}
  \begin{columns}
    \begin{column}{0.5\textwidth}
      \textbf{1. Crypto-Ransomware}
      \begin{itemize}
        \item Verschlüsselt Benutzerdateien
        \item Verwendet AES, RSA
        \item System bleibt funktional
        \item Wiederherstellung oft unmöglich
      \end{itemize}
    \end{column}
    \begin{column}{0.5\textwidth}
      \textbf{2. Locker-Ransomware}
      \begin{itemize}
        \item Sperrt Systemzugriff
        \item Bildschirmsperre
        \item Keine Datenverschlüsselung
        \item Einfacher zu beheben
      \end{itemize}
    \end{column}
  \end{columns}
\end{frame}

% ==========================================================
% NEUE SEKTION: TECHNISCHE IMPLEMENTIERUNG
% ==========================================================
\section{Case Study: Edu-Ransomware Architektur}

\begin{frame}{Technische Übersicht}
  Um die Theroie besser verständlich zu machen, haben wir eine Simulation enwickelt.
  
  \vspace{1em}
  \begin{columns}
    \begin{column}{0.5\textwidth}
        \textbf{Komponente A: Agent (Victim)}
        \begin{itemize}
            \item Sprache: \textbf{Rust} (High Performance, Memory Safety)
            \item Cross-Platform (Windows/Linux/macOS)
            \item Features: AES-256 Verschlüsselung, Sandbox-Evasion, Persistenz
        \end{itemize}
    \end{column}
    \begin{column}{0.5\textwidth}
        \textbf{Komponente B: C2-Server (Attacker)}
        \begin{itemize}
            \item Sprache: \textbf{Python}
            \item Multi-Threaded TCP Server
            \item Steuerung, Exfiltration (Data Theft), Key-Management
        \end{itemize}
    \end{column}
  \end{columns}
\end{frame}

\begin{frame}{Modulare Architektur (Agent)}
  \vspace{-0.5em}
  Der Rust-Agent bildet die Kill-Chain durch isolierte, spezialisierte Module ab:
  \vspace{1em}

  \begin{columns}[t]
    % LINKE SPALTE
    \begin{column}{0.48\textwidth}
      \begin{block}{1. Defense Evasion (\texttt{evasion.rs})}
        \footnotesize
        \begin{itemize}
            \item \textbf{Sandbox-Erkennung:} Prüft RAM (<3GB) und CPU-Cores (<2).
            \item \textbf{Reaktion:} Sofortiger Prozessabbruch zur Tarnung vor Analysten.
        \end{itemize}
      \end{block}
      
      \vspace{0.5em}
      
      \begin{block}{3. Impact Engine (\texttt{crypto.rs})}
        \footnotesize
        \begin{itemize}
            \item \textbf{Algorithmus:} AES-256-CTR (Stream Cipher).
            \item \textbf{Atomic Operations:} Rename zu \texttt{.locked} verhindert Datenverlust bei Absturz.
        \end{itemize}
      \end{block}
    \end{column}

    % RECHTE SPALTE
    \begin{column}{0.48\textwidth}
      \begin{block}{2. C2 Channel (\texttt{network.rs})}
        \footnotesize
        \begin{itemize}
            \item \textbf{Protokoll:} TCP-Loop mit Retry-Logik.
            \item \textbf{Commands:} Parsing von \texttt{shell}, \texttt{exfil}, \texttt{encrypt}.
        \end{itemize}
      \end{block}
      
      \vspace{0.5em}
      
      \begin{block}{4. Extortion Logic (\texttt{extortion.rs})}
        \footnotesize
        \begin{itemize}
            \item \textbf{User Notification:} Ersetzt "Panic Mode".
            \item \textbf{Mechanismus:} Browser-Loop öffnet Lösegeldforderung alle 5s (Psychologischer Druck).
        \end{itemize}
      \end{block}
    \end{column}
  \end{columns}
\end{frame}

\begin{frame}{Delivery Infrastruktur}
    \textbf{Intelligente Verteilung:}
    \begin{itemize}
        \item Python-basierter Webserver als Delivery-Point.
        \item \textbf{Smart Endpoint:} Erkennt Betriebssystem des Opfers via User-Agent Header.
        \item Liefert automatisch passendes Binary (`.exe` für Windows, Elf für Linux).
    \end{itemize}
    
    \vspace{0.5em}
    \textbf{Infektionsvektoren:}
    \begin{itemize}
        \item \textbf{Phishing PDF:} Verschwommene Rechnung mit Fake-Sicherheitshinweis.
        \item \textbf{Drive-by-Download:} Fake-Webseiten (Gaming, Gewinnspiel, Sicherheitswarnung).
    \end{itemize}
\end{frame}

% ==========================================================
% SEKTION 5: PHASEN EINES ANGRIFFS (MIT DEMO)
% ==========================================================
\section{Die Phasen eines Angriffs}

\begin{frame}{Phase 1: Distribution \& Infection}
  \textbf{Theorie:}
  \begin{itemize}
    \item Phishing (E-Mail Anhänge, Links)
    \item Drive-by-Downloads
  \end{itemize}
  
  \vspace{1em}
  
  \begin{demobox}{Live Demo: Initial Access Szenario}
    \textbf{Szenario:} Ein Mitarbeiter erhält eine E-Mail mit dem Anhang "Rechnung\_Dez.pdf".
    
    \begin{itemize}
        \item \textbf{Der Köder:} Das PDF enthält ein unscharfes Bild, das wie eine echte Rechnung aussieht.
        \item \textbf{Der Trick:} Ein Button "Inhalt entschlüsseln" suggeriert ein technisches Problem (Social Engineering).
        \item \textbf{Das Ergebnis:} Beim Klick liefert der Smart-Server die Malware aus und infiziert das System.
    \end{itemize}
  \end{demobox}
  
\end{frame}

\begin{frame}{Phase 2: Execution \& Evasion}

  \textbf{Theorie:} Ransomware installiert sich, prüft auf Sandboxes und etabliert Persistenz.
  
  \vspace{0.3em}
  
  \begin{demobox}{Projekt-Einblick: Evasion Module}
    Der Agent prüft beim Start:
    \begin{itemize}
        \item Ist RAM < 3GB?
        \item Sind weniger als 2 CPU-Kerne verfügbar?
    \end{itemize}
    Falls ja: \alert{Sofortiger Abbruch} mit gefälschter Fehlermeldung, um Analysten zu täuschen.
  \end{demobox}
\end{frame}

\begin{frame}{Phase 3: C2 \& Exfiltration}
  \textbf{Theorie:} Aufbau der Kommunikation, Nachladen von Befehlen, Datendiebstahl.
  
  \vspace{0.5em}
  
  \begin{demobox}{Live Demo: Attacker Control}
    Wir wechseln zum Angreifer-Terminal (C2):
    \begin{itemize}
        \item \texttt{[+] New Victim Connected: ID 1}
        \item Angreifer nutzt \texttt{shell}-Befehle zur Erkundung (`ls`, `ps`).
        \item \textbf{Double Extortion:} Befehl \texttt{exfil secret.pdf} stiehlt Daten vor der Verschlüsselung.
    \end{itemize}
  \end{demobox}
\end{frame}

\begin{frame}{Phase 4: Encryption (Impact)}
    \textbf{Theorie:} Starke Verschlüsselung, Löschung von Backups.
      
    \vspace{0.5em}
      
    \begin{demobox}{Live Demo: The Panic Mode}
      Angreifer sendet \texttt{encrypt}. Auswirkungen auf Opfer-PC:
      \begin{itemize}
        \item Dateien erhalten Endung \texttt{.locked}.
        \item \textbf{Visuell:} Browser öffnet die Lösegeldforderung (Stress-Erzeugung).
        \item Log-File zeigt Verschlüsselungsprozess in Echtzeit.
      \end{itemize}
    \end{demobox}
\end{frame}

\begin{frame}{Phase 5: Decryption (Recovery)}
    \textbf{Szenario:} Das Lösegeld wurde gezahlt (in der Simulation).
    
    \vspace{1em}
    \begin{itemize}
        \item Angreifer sendet Befehl \texttt{decrypt}.
        \item Agent nutzt den symmetrischen Key (AES-CTR), um Dateien wiederherzustellen.
        \item \texttt{.locked} Dateien verschwinden, Originale sind wieder da.
    \end{itemize}
\end{frame}


% Sektion 6: Gegenmaßnahmen
\section{Gegenma{\ss}nahmen}

\begin{frame}{Prävention \& Detektion}
  \textbf{Technische Ma{\ss}nahmen:}
  \begin{itemize}
    \item \textbf{Backups}: 3-2-1-Regel
    \item \textbf{EDR / Antivirus}: Verhaltensanalyse
  \end{itemize}
  
  \vspace{0.5em}
  
  \begin{exampleblock}{Projekt-Reflektion: Warum Evasion?}
    \vspace{0.1cm}
    Herkömmliche AV-Systeme scannen oft statisch.
    Unsere Ransomware umgeht dies durch:
    \begin{itemize}
        \item Dynamisches Nachladen (keine Signatur beim Download des PDF).
        \item Sandbox-Checks (verhindert Analyse in der Cloud).
        \item Nutzung seltener Sprachen (Rust binaries sind oft schwerer zu analysieren als C++).
    \end{itemize}
  \end{exampleblock}
\end{frame}

\begin{frame}{Detektion und Reaktion}
  \textbf{Detektionsansätze:}
  \begin{itemize}
    \item \textbf{Statische Analyse}: Untersuchung ohne Ausführung
    \item \textbf{Dynamische Analyse}: Verhaltensbeobachtung in Sandbox
    \item \textbf{Machine Learning}: Erkennung unbekannter Varianten
    \item \textbf{Netzwerkbasierte Detektion}: Anomalie-Erkennung im Traffic
  \end{itemize}
  
  \vspace{0.3em}
  
  \textbf{Wichtige Merkmale:}
  \begin{itemize}
    \item API-Aufrufe und Systemverhalten
    \item Datei-/Verzeichnisaktivitäten
    \item Netzwerkverkehrsmuster
    \item Verschlüsselungsoperationen
  \end{itemize}
\end{frame}

\begin{frame}{Incident Response}
  \textbf{Sofortma{\ss}nahmen bei Verdacht:}
  \begin{enumerate}
    \item \alert{Isolation} betroffener Systeme
    \item Identifikation des Ransomware-Stamms
    \item Bewertung der Schadenausbreitung
    \item Benachrichtigung relevanter Stellen
  \end{enumerate}
  
  \vspace{0.3em}
  
  \textbf{Wiederherstellung:}
  \begin{itemize}
    \item Wiederherstellung aus Backups (wenn möglich)
    \item Key-Escrow-Mechanismen
    \item Forensische Analyse
    \item Systemhärtung vor Wiederinbetriebnahme
  \end{itemize}
  
  \vspace{0.3em}
  
  \alert{Zahlungsempfehlung}: Keine Lösegeldzahlung (keine Garantie, finanziert weitere Angriffe)
\end{frame}

% Sektion 7: Forschung & Ausblick
\section{Forschung \& Ausblick}
% Inhalte bleiben gleich (REDFISH, Zero-Loss Recovery, etc.)
\begin{frame}{Aktuelle Forschungslandschaft}
  \textbf{Fokusgebiete:}
  \begin{itemize}
    \item \textbf{72,8\%} der Studien: Ransomware-Detektion
    \item Machine Learning dominiert als Detektionsmethode
    \item Hohe Genauigkeit bei unbekannten Varianten
  \end{itemize}
  
  \vspace{0.5em}
  
  \textbf{Innovative Ansätze:}
  \begin{itemize}
    \item \textbf{REDFISH}: Netzwerkbasierte Früherkennung
    \begin{itemize}
      \item Monitoring von SMB-Protokoll
      \item Erkennung von Verhaltensmustern (schnelles Lesen/Schreiben/Löschen)
      \item \alert{99\% Erkennung vor Verlust von 10 Dateien}
    \end{itemize}
    \item Near-Zero-Loss-Szenario durch Traffic-Wiederherstellung
    \item Moving Target Defense (MTD)
  \end{itemize}
\end{frame}

\begin{frame}{Zero-Loss Recovery: REDFISH}
  \textbf{Funktionsweise:}
  \begin{itemize}
    \item Analyse des Netzwerk-Traffics zu Freigaben
    \item Erkennung anomaler Zugriffsmuster
    \item Aufzeichnung der Dateiinhalte
    \item Automatische Alarmierung
  \end{itemize}
  
  \vspace{0.5em}
  
  \textbf{Vorteile:}
  \begin{itemize}
    \item Früherkennung (99\% bei <10 Dateien)
    \item Wiederherstellung ohne Backups
    \item Minimaler Datenverlust
  \end{itemize}
\end{frame}

% Sektion 8: Fazit
\section{Fazit}

\begin{frame}{Zusammenfassung}
  \textbf{Takeaways aus der Simulation:}
  \begin{itemize}
    \item Angriffe sind heute hochgradig automatisiert (Delivery Server).
    \item Social Engineering (PDF) ist oft effektiver als technische Exploits.
    \item Ransomware ist nicht nur Verschlüsselung, sondern auch Datendiebstahl.
  \end{itemize}
  
  \vspace{0.5em}
  \alert{Prävention und Awareness sind der beste Schutz.}
\end{frame}

\begin{frame}{Zukünftige Herausforderungen}
  \textbf{Technologische Entwicklungen:}
  \begin{itemize}
    \item \textbf{KI in Ransomware}: Adaptives, intelligentes Verhalten
    \item \textbf{Quantencomputing}: Bedrohung für aktuelle Verschlüsselung
    \item \textbf{IoT-Ransomware}: Neue Angriffsflächen
    \item \textbf{Cloud-native Ransomware}: Angriffe auf Cloud-Infrastrukturen
  \end{itemize}
  
  \vspace{0.5em}
  
  \textbf{Forschungsbedarf:}
  \begin{itemize}
    \item Echtzeit-Schutz und Zero-Day-Erkennung
    \item Post-Quantum-Kryptographie
    \item Automatisierte Incident Response
    \item Internationale Zusammenarbeit bei Strafverfolgung
  \end{itemize}
\end{frame}


\begin{frame}{Quellen}
  \begin{thebibliography}{99}
    \bibitem{age} \textit{The Age of Ransomware: A Survey on the Evolution, Taxonomy, and Research Directions} (Razulla et al.)
    \bibitem{enisa} ENISA (2024): \textit{Threat Landscape Report 2024}
    \bibitem{project} \textit{Edu-Ransomware Repository} (GitHub, 2026) - Eigene Entwicklung
  \end{thebibliography}
\end{frame}

\end{document}
