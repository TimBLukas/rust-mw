\documentclass[aspectratio=169]{beamer}

% ============================================================
% THEME UND PACKAGES
% ============================================================
\usetheme{metropolis}
\usepackage[utf8]{inputenc}
\usepackage[T1]{fontenc}
\usepackage[ngerman]{babel}
\usepackage{graphicx}
\usepackage{booktabs}
\usepackage{xcolor}
\usepackage{tikz}
\usepackage{pgfplots}
\pgfplotsset{compat=1.18}
\usepackage[edges]{forest}
\usetikzlibrary{shapes,arrows,positioning,fit,backgrounds,calc}
\usepackage{tcolorbox}
\tcbuselibrary{skins,breakable}
\usepackage{setspace}
\usepackage{microtype}

% ============================================================
% FARBSCHEMA
% ============================================================
\definecolor{primarydark}{RGB}{15,23,42}
\definecolor{primaryblue}{RGB}{30,64,175}
\definecolor{accentcyan}{RGB}{6,182,212}
\definecolor{accentorange}{RGB}{251,146,60}
\definecolor{accentred}{RGB}{239,68,68}
\definecolor{accentgreen}{RGB}{34,197,94}
\definecolor{accentpurple}{RGB}{168,85,247}
\definecolor{lightgray}{RGB}{241,245,249}
\definecolor{textgray}{RGB}{71,85,105}

\definecolor{chartcolor1}{RGB}{239,68,68}
\definecolor{chartcolor2}{RGB}{59,130,246}
\definecolor{chartcolor3}{RGB}{34,197,94}
\definecolor{chartcolor4}{RGB}{251,191,36}
\definecolor{chartcolor5}{RGB}{168,85,247}
\definecolor{chartcolor6}{RGB}{251,146,60}

\definecolor{rustcolor}{RGB}{183,65,14}

% ============================================================
% BEAMER THEME KONFIGURATION
% ============================================================
\setbeamercolor{normal text}{fg=primarydark,bg=white}
\setbeamercolor{alerted text}{fg=accentred}
\setbeamercolor{example text}{fg=accentgreen}
\setbeamercolor{frametitle}{bg=primarydark,fg=white}
\setbeamercolor{title separator}{fg=accentcyan}
\setbeamercolor{progress bar}{fg=accentcyan,bg=primarydark!20}
\setbeamercolor{block title}{fg=white,bg=primaryblue}
\setbeamercolor{block body}{fg=primarydark,bg=lightgray}
\setbeamercolor{block title example}{fg=white, bg=rustcolor}
\setbeamercolor{block body example}{bg=rustcolor!8}
\setbeamercolor{itemize item}{fg=primaryblue}
\setbeamercolor{enumerate item}{fg=primaryblue}

\metroset{
  titleformat=regular,
  sectionpage=progressbar,
  numbering=fraction,
  progressbar=frametitle,
  block=fill
}

% ============================================================
% CUSTOM BOXES
% ============================================================
\newtcolorbox{demobox}[1]{
  enhanced,
  colback=white,
  colframe=primaryblue,
  colbacktitle=primaryblue,
  coltitle=white,
  coltext=primarydark,
  title={\textbf{#1}},
  fonttitle=\sffamily\bfseries\small,
  fontupper=\sffamily\footnotesize,
  arc=3pt,
  boxrule=0pt,
  leftrule=3pt,
  toptitle=4pt,
  bottomtitle=4pt,
  left=8pt, right=8pt, top=4pt, bottom=6pt,
  before skip=6pt,
  after skip=6pt
}

\newtcolorbox{infobox}[1]{
  enhanced,
  colback=accentcyan!5,
  colframe=accentcyan,
  coltext=primarydark,
  title={\textbf{#1}},
  coltitle=accentcyan,
  fonttitle=\sffamily\bfseries\small,
  fontupper=\sffamily\footnotesize,
  arc=3pt,
  boxrule=0pt,
  leftrule=3pt,
  left=8pt, right=8pt, top=4pt, bottom=6pt,
  before skip=6pt,
  after skip=6pt
}

\newtcolorbox{warnbox}[1]{
  enhanced,
  colback=accentorange!8,
  colframe=accentorange,
  coltext=primarydark,
  title={\textbf{#1}},
  coltitle=accentorange!80!black,
  fonttitle=\sffamily\bfseries\small,
  fontupper=\sffamily\footnotesize,
  arc=3pt,
  boxrule=0pt,
  leftrule=3pt,
  left=8pt, right=8pt, top=4pt, bottom=6pt,
  before skip=6pt,
  after skip=6pt
}

% ============================================================
% TYPOGRAPHY
% ============================================================
\setbeamerfont{title}{size=\LARGE,series=\bfseries}
\setbeamerfont{subtitle}{size=\normalsize}
\setbeamerfont{frametitle}{size=\large,series=\bfseries}
\setbeamerfont{block title}{size=\small,series=\bfseries}

\setlength{\leftmargini}{1em}
\setlength{\leftmarginii}{0.8em}

\newcommand{\code}[1]{\texttt{\textcolor{primaryblue}{#1}}}

% ============================================================
% TITELSEITE
% ============================================================
\title{Ransomware}
\subtitle{Anatomie, Netzwerkerkennung und Zero-Loss Recovery}
\author{Ihr Name}
\date{\today}
\institute{Ihre Institution}

\begin{document}

% ============================================================
% TITELFOLIE
% ============================================================
{
\setbeamertemplate{background}{
  \begin{tikzpicture}[overlay,remember picture]
    \shade[top color=primarydark,bottom color=primaryblue!50!primarydark] 
      (current page.south west) rectangle (current page.north east);
    \draw[accentcyan,line width=2pt] 
      ([yshift=-0.5cm]current page.center) ++(-5,0) -- ++(10,0);
  \end{tikzpicture}
}
\setbeamercolor{title}{fg=white}
\setbeamercolor{subtitle}{fg=accentcyan}
\setbeamercolor{author}{fg=white!80}
\setbeamercolor{institute}{fg=white!60}
\setbeamercolor{date}{fg=white!60}
\begin{frame}[plain]
  \vspace{2cm}
  \begin{center}
    {\usebeamerfont{title}\usebeamercolor[fg]{title}\inserttitle}\\[0.6em]
    {\usebeamerfont{subtitle}\usebeamercolor[fg]{subtitle}\insertsubtitle}\\[1.5em]
    {\small\usebeamercolor[fg]{author}\insertauthor}\\[0.2em]
    {\footnotesize\usebeamercolor[fg]{institute}\insertinstitute}\\[0.2em]
    {\footnotesize\usebeamercolor[fg]{date}\insertdate}
  \end{center}
\end{frame}
}

% ============================================================
% AGENDA
% ============================================================
\begin{frame}{Agenda}
  \begin{columns}[T]
    \begin{column}{0.48\textwidth}
      \tableofcontents[sections={1-4}]
    \end{column}
    \begin{column}{0.48\textwidth}
      \tableofcontents[sections={5-8}]
    \end{column}
  \end{columns}
\end{frame}

% ============================================================
% SEKTION 1: ENTWICKLUNG & GESCHICHTE
% ============================================================
\section{Entwicklung \& Geschichte}

\begin{frame}{Historischer Abriss}
  \begin{columns}[T]
    \begin{column}{0.48\textwidth}
      \textbf{Meilensteine:}
      \begin{itemize}
        \item \textbf{1989}: AIDS Trojan -- erster dokumentierter Ransomware-Angriff
        \item \textbf{2005}: Wiederaufleben durch Internet und Kryptowährungen
        \item \textbf{2017}: WannaCry-Ausbruch mit globaler Aufmerksamkeit
      \end{itemize}
    \end{column}
    \begin{column}{0.48\textwidth}
      \begin{figure}
        \centering
        \includegraphics[width=\textwidth]{images/historical_timeline.png}
        \caption{\footnotesize Entwicklung von Ransomware}
      \end{figure}
    \end{column}
  \end{columns}
\end{frame}

% ============================================================
% SEKTION 2: AKTUELLE BEDROHUNGSLAGE
% ============================================================
\section{Aktuelle Bedrohungslage}

\begin{frame}{Finanzielle Auswirkungen}
  \begin{columns}[T]
    \begin{column}{0.35\textwidth}
      \textbf{Geschätzte Weltweite Schäden}
      \vspace{0.5em}
      
      \begin{tabular}{ll}
        2017: & \textbf{5 Mrd. USD}\\
        2024: & \textbf{42 Mrd. USD}\\
      \end{tabular}
      
      \vspace{0.5em}
      \textcolor{accentred}{\textbf{+740\%}} Anstieg in 7 Jahren
    \end{column}
    \begin{column}{0.62\textwidth}
      \begin{tikzpicture}
        \begin{axis}[
          width=\textwidth,
          height=0.6\textheight,
          xlabel={\footnotesize Jahr},
          ylabel={\footnotesize Schaden (Mrd. USD)},
          ymin=0, ymax=48,
          xtick={2015,2017,2019,2021,2024},
          ytick={0,10,20,30,40},
          grid=both,
          grid style={line width=.1pt, draw=lightgray},
          x tick label style={/pgf/number format/1000 sep={},font=\scriptsize},
          y tick label style={font=\scriptsize},
        ]
        \addplot[fill=chartcolor1!30,draw=chartcolor1,line width=1.5pt,mark=*,mark options={fill=chartcolor1,scale=0.8}] coordinates {
          (2015,0.325)(2017,5)(2019,11.5)(2021,20)(2024,42)
        } \closedcycle;
        \end{axis}
      \end{tikzpicture}
    \end{column}
  \end{columns}
\end{frame}

\begin{frame}{ENISA Threat Landscape 2024}
  \begin{columns}[T]
    \begin{column}{0.35\textwidth}
      \textbf{Top-Bedrohungen EU:}
      \begin{enumerate}
        \item DDoS/RDoS (46,31\%)
        \item \alert{Ransomware (27,33\%)}
        \item Data Breaches (15,87\%)
        \item Social Engineering (5,37\%)
        \item Malware (2,45\%)
        \item Zero-Day (0,11\%)
      \end{enumerate}
      
      \vspace{0.3em}
      \small\textbf{Ransomware: Zweitgrößte Bedrohung}
    \end{column}
    \begin{column}{0.62\textwidth}
      \begin{tikzpicture}
        \begin{axis}[
          width=\textwidth,
          height=0.65\textheight,
          xbar,
          xlabel={\footnotesize Vorfälle (Tausend)},
          symbolic y coords={Zero-Day, Malware, Social Eng., Data, Ransomware, DDoS},
          ytick=data,
          nodes near coords,
          nodes near coords style={font=\scriptsize},
          xmin=0, xmax=2.8,
          bar width=0.4cm,
          y tick label style={font=\scriptsize},
          x tick label style={font=\scriptsize},
          enlarge y limits=0.12,
        ]
        \addplot[fill=chartcolor1,draw=chartcolor1!80!black] coordinates {
          (0.01,Zero-Day)(0.13,Malware)(0.29,Social Eng.)(0.84,Data)(1.45,Ransomware)(2.46,DDoS)
        };
        \end{axis}
      \end{tikzpicture}
    \end{column}
  \end{columns}
\end{frame}

% ============================================================
% SEKTION 3: ZIELE & MOTIVATION
% ============================================================
\section{Ziele \& Motivation}

\begin{frame}{Am häufigsten betroffene Sektoren}
  \begin{columns}[T]
    \begin{column}{0.35\textwidth}
      \textbf{Top 5 Sektoren:}
      \begin{enumerate}
        \item Manufacturing (21,78\%)
        \item Retail (11,06\%)
        \item ICT Services (9,18\%)
        \item Business Services (7,14\%)
        \item Banking/Finance (6,94\%)
      \end{enumerate}
      
      \vspace{0.3em}
      \small Industrie und Fertigung am stärksten betroffen
    \end{column}
    \begin{column}{0.62\textwidth}
      \begin{tikzpicture}
        \begin{axis}[
          width=\textwidth,
          height=0.75\textheight,
          xbar,
          xlabel={\footnotesize Anteil (\%)},
          symbolic y coords={Water, Energy, Public Admin, Transport, Education, Banking, Business, ICT, Other, Retail, Manufacturing},
          ytick=data,
          nodes near coords,
          nodes near coords style={font=\tiny},
          xmin=0, xmax=25,
          bar width=0.25cm,
          y tick label style={font=\tiny},
          x tick label style={font=\scriptsize},
          enlarge y limits=0.05,
        ]
        \addplot[fill=chartcolor2,draw=chartcolor2!80!black] coordinates {
          (0.37,Water)(2.78,Energy)(3.58,Public Admin)(3.97,Transport)(4.92,Education)(6.94,Banking)(7.14,Business)(9.18,ICT)(10.52,Other)(11.06,Retail)(21.78,Manufacturing)
        };
        \end{axis}
      \end{tikzpicture}
    \end{column}
  \end{columns}
\end{frame}

\begin{frame}{Motivation der Angreifer}
  \textbf{Warum diese Ziele?}
  \vspace{0.5em}
  
  \begin{columns}[T]
    \begin{column}{0.48\textwidth}
      \begin{block}{Finanzielle Motive}
        \small Zahlungsbereitschaft bei kritischen Systemen
      \end{block}
      \vspace{0.3em}
      \begin{block}{Kritische Infrastrukturen}
        \small Hoher Druck durch Ausfallkosten
      \end{block}
    \end{column}
    \begin{column}{0.48\textwidth}
      \begin{block}{Sensible Daten}
        \small Erpressungspotenzial durch Datenleaks
      \end{block}
      \vspace{0.3em}
      \begin{block}{Geopolitische Faktoren}
        \small Destabilisierung und Spionage
      \end{block}
    \end{column}
  \end{columns}
  
  \vspace{0.5em}
  \begin{center}
    \alert{\textbf{Zunehmende Professionalisierung und Organisation}}
  \end{center}
\end{frame}

% ============================================================
% RANSOMWARE AS A BUSINESS
% ============================================================

\begin{frame}{Warum die Bedrohung bleibt: Die Ökonomie}
  \textbf{Ransomware-as-a-Service (RaaS)} ist eine hochprofessionelle Schattenwirtschaft mit Milliardenumsätzen:
  \vspace{0.5em}

  \begin{columns}[T]
    \begin{column}{0.32\textwidth}
      \begin{block}{1. Access Provider}
        \footnotesize
        Spezialisten, die Zugänge zu Netzwerken potenzieller Opfer verkaufen.
      \end{block}
    \end{column}
    \begin{column}{0.32\textwidth}
      \begin{block}{2. RaaS Provider}
        \footnotesize
        Entwickler der Malware. Bieten C2-Infrastruktur und Support gegen Provision.
      \end{block}
    \end{column}
    \begin{column}{0.32\textwidth}
      \begin{block}{3. Affiliates}
        \footnotesize
        Kaufen Zugänge, mieten Malware und führen den Angriff durch.
      \end{block}
    \end{column}
  \end{columns}
\end{frame}

\begin{frame}{Professionalisierungsgrad: Beispiel LockBit}
  Die Qualität der Services übertrifft teilweise legale SaaS-Anbieter.
  
  \vspace{0.5em}
  
  \begin{demobox}{Beispiel: LockBit 3.0 Bug-Bounty}
    LockBit hat ein eigenes Prämienprogramm für Hinweise zur Verbesserung ihrer Schadsoftware aufgelegt:
    \vspace{0.2em}
    
    \textit{"Locker Bugs: Any errors during encryption ... that lead to corrupted files or to the possibility of decrypting files without getting a decryptor."}
  \end{demobox}

  \vspace{0.3em}
  
  \begin{warnbox}{Wirtschaftliches Risiko}
    Im Gegensatz zu Banken ("Too big to fail") gilt für Ransomware-Gruppen: \textbf{"Too big to prevail"}. Wer zu groß wird, zieht zu viel Aufmerksamkeit der Strafverfolger auf sich.
  \end{warnbox}
\end{frame}

% ============================================================
% SEKTION 4: ARTEN VON RANSOMWARE
% ============================================================
\section{Arten von Ransomware}

\begin{frame}{Ransomware Taxonomy}
  \tikzset{
    mybox/.style={font=\scriptsize\sffamily\bfseries,draw=primaryblue!50, line width=0.6pt, inner sep=3pt, rounded corners=2pt, align=center},
    greenbox/.style={mybox, fill=primaryblue!15, text=primarydark, text width=2cm, minimum height=0.8cm},
    bluebox/.style={mybox, fill=accentcyan!15, text=primarydark, text width=2cm, font=\tiny\sffamily}
  }

  \centering
  \resizebox{0.95\textwidth}{!}{%
    \begin{forest}
      for tree={edge={draw, primaryblue!50, line width=0.6pt}, s sep=0.2cm, l sep=0.4cm}
      [, phantom, forked edges
        [Types, greenbox, for children={folder, grow'=0} [Crypto, bluebox][Locker, bluebox][Scareware, bluebox]]
        [Targets, greenbox, for children={folder, grow'=0} [Platforms, bluebox][Industries, bluebox]]
        [Infection vectors, greenbox, for children={folder, grow'=0} [Windows RDP, bluebox][Phishing, bluebox][Software vuln., bluebox]]
      ]
    \end{forest}
  }

  \vspace{0.3cm}

  \resizebox{0.95\textwidth}{!}{%
    \begin{forest}
      for tree={edge={draw, primaryblue!50, line width=0.6pt}, s sep=0.2cm, l sep=0.4cm}
      [, phantom, forked edges
        [Comm. with C\&C, greenbox, for children={folder, grow'=0} [Hard-coded IP, bluebox][DGA, bluebox]]
        [Access to enc. key, greenbox, for children={folder, grow'=0} [Hard-coded, bluebox][Via C\&C, bluebox]]
        [Malicious action, greenbox, for children={folder, grow'=0} [Encryption, bluebox][Locking, bluebox]]
      ]
    \end{forest}
  }
  
  \vspace{0.2cm}
  
  \begin{exampleblock}{Projekt-Einblick: Edu-Ransomware}
    \footnotesize
    Unsere Ransomware ist eine klassische \textbf{Crypto-Ransomware} mit \textbf{Double Extortion}.
  \end{exampleblock}
\end{frame}

\begin{frame}{Grundlegende Typen}
  \begin{columns}[T]
    \begin{column}{0.48\textwidth}
      \begin{block}{1. Crypto-Ransomware}
        \small
        \begin{itemize}
          \item Verschlüsselt Benutzerdateien
          \item Verwendet AES, RSA
          \item System bleibt funktional
          \item Wiederherstellung oft unmöglich
        \end{itemize}
      \end{block}
    \end{column}
    \begin{column}{0.48\textwidth}
      \begin{block}{2. Locker-Ransomware}
        \small
        \begin{itemize}
          \item Sperrt Systemzugriff
          \item Bildschirmsperre
          \item Keine Datenverschlüsselung
          \item Einfacher zu beheben
        \end{itemize}
      \end{block}
    \end{column}
  \end{columns}
\end{frame}

% ============================================================
% SEKTION: CASE STUDY
% ============================================================
\section{Case Study: Edu-Ransomware Architektur}

\begin{frame}{Technische Übersicht}
  Um die Theorie besser verständlich zu machen, haben wir eine Simulation entwickelt.
  
  \vspace{0.5em}
  \begin{columns}[T]
    \begin{column}{0.48\textwidth}
      \begin{block}{Komponente A: Agent (Victim)}
        \footnotesize
        \begin{itemize}
          \item Sprache: \textbf{Rust} (High Performance)
          \item Cross-Platform (Win/Linux/macOS)
          \item AES-256, Sandbox-Evasion, Persistenz
        \end{itemize}
      \end{block}
    \end{column}
    \begin{column}{0.48\textwidth}
      \begin{block}{Komponente B: C2-Server (Attacker)}
        \footnotesize
        \begin{itemize}
          \item Sprache: \textbf{Python}
          \item Multi-Threaded TCP Server
          \item Steuerung, Exfiltration, Key-Mgmt
        \end{itemize}
      \end{block}
    \end{column}
  \end{columns}
\end{frame}

\begin{frame}{Modulare Architektur (Agent)}
  Der Rust-Agent bildet die Kill-Chain durch spezialisierte Module ab:
  \vspace{0.3em}

  \begin{columns}[T]
    \begin{column}{0.48\textwidth}
      \begin{block}{1. Defense Evasion (\code{evasion.rs})}
        \footnotesize
        \begin{itemize}
          \item Sandbox-Erkennung: RAM <3GB, CPU <2
          \item Reaktion: Sofortiger Prozessabbruch
        \end{itemize}
      \end{block}
      
      \vspace{0.2em}
      
      \begin{block}{3. Impact Engine (\code{crypto.rs})}
        \footnotesize
        \begin{itemize}
          \item Algorithmus: AES-256-CTR
          \item Atomic Operations mit \code{.locked}
        \end{itemize}
      \end{block}
    \end{column}

    \begin{column}{0.48\textwidth}
      \begin{block}{2. C2 Channel (\code{network.rs})}
        \footnotesize
        \begin{itemize}
          \item Protokoll: TCP-Loop mit Retry
          \item Commands: shell, exfil, encrypt
        \end{itemize}
      \end{block}
      
      \vspace{0.2em}
      
      \begin{block}{4. Extortion Logic (\code{extortion.rs})}
        \footnotesize
        \begin{itemize}
          \item User Notification
          \item Browser-Loop alle 5s
        \end{itemize}
      \end{block}
    \end{column}
  \end{columns}
\end{frame}

\begin{frame}{Delivery Infrastruktur}
  \begin{columns}[T]
    \begin{column}{0.48\textwidth}
      \begin{block}{Intelligente Verteilung}
        \footnotesize
        \begin{itemize}
          \item Python-basierter Webserver
          \item Erkennt OS via User-Agent
          \item Liefert passendes Binary
        \end{itemize}
      \end{block}
    \end{column}
    \begin{column}{0.48\textwidth}
      \begin{block}{Infektionsvektoren}
        \footnotesize
        \begin{itemize}
          \item Phishing PDF: Verschwommene Rechnung
          \item Drive-by-Download: Fake-Webseiten
        \end{itemize}
      \end{block}
    \end{column}
  \end{columns}
\end{frame}

% ============================================================
% SEKTION 5: PHASEN EINES ANGRIFFS
% ============================================================
\section{Die Phasen eines Angriffs}

\begin{frame}{Phase 1: Distribution \& Infection}
  \textbf{Theorie:} Phishing (E-Mail Anhänge, Links), Drive-by-Downloads
  
  \vspace{0.5em}
  
  \begin{demobox}{Live Demo: Initial Access Szenario}
    \textbf{Szenario:} Mitarbeiter erhält E-Mail mit "Rechnung\_Dez.pdf"
    \begin{itemize}
      \item \textbf{Köder:} PDF enthält unscharfes Bild einer Rechnung
      \item \textbf{Trick:} Button "Inhalt entschlüsseln" (Social Engineering)
      \item \textbf{Ergebnis:} Smart-Server liefert Malware aus
    \end{itemize}
  \end{demobox}
\end{frame}

\begin{frame}{Phase 2: Execution \& Evasion}
  \textbf{Theorie:} Ransomware installiert sich, prüft auf Sandboxes, etabliert Persistenz.
  
  \vspace{0.5em}
  
  \begin{demobox}{Projekt-Einblick: Evasion Module}
    Der Agent prüft beim Start:
    \begin{itemize}
      \item Ist RAM < 3GB?
      \item Sind weniger als 2 CPU-Kerne verfügbar?
    \end{itemize}
    Falls ja: \alert{Sofortiger Abbruch} mit gefälschter Fehlermeldung.
  \end{demobox}
\end{frame}

\begin{frame}{Phase 3: C2 \& Exfiltration}
  \textbf{Theorie:} Aufbau der Kommunikation, Nachladen von Befehlen, Datendiebstahl.
  
  \vspace{0.5em}
  
  \begin{demobox}{Live Demo: Attacker Control}
    Wir wechseln zum Angreifer-Terminal (C2):
    \begin{itemize}
      \item \code{[+] New Victim Connected: ID 1}
      \item Angreifer nutzt \code{shell}-Befehle zur Erkundung
      \item \textbf{Double Extortion:} \code{exfil secret.pdf} stiehlt Daten
    \end{itemize}
  \end{demobox}
\end{frame}

\begin{frame}{Phase 4: Encryption (Impact)}
  \textbf{Theorie:} Starke Verschlüsselung, Löschung von Backups.
      
  \vspace{0.5em}
      
  \begin{demobox}{Live Demo: The Panic Mode}
    Angreifer sendet \code{encrypt}. Auswirkungen auf Opfer-PC:
    \begin{itemize}
      \item Dateien erhalten Endung \code{.locked}
      \item Browser öffnet Lösegeldforderung (Stress)
      \item Log-File zeigt Verschlüsselung in Echtzeit
    \end{itemize}
  \end{demobox}
\end{frame}

\begin{frame}{Phase 5: Decryption (Recovery)}
  \textbf{Szenario:} Das Lösegeld wurde gezahlt (in der Simulation).
    
  \vspace{0.5em}
  
  \begin{itemize}
    \item Angreifer sendet Befehl \code{decrypt}
    \item Agent nutzt symmetrischen Key (AES-CTR)
    \item \code{.locked} Dateien verschwinden, Originale sind wieder da
  \end{itemize}
\end{frame}

% ============================================================
% SEKTION 6: GEGENMAßNAHMEN
% ============================================================
\section{Gegenma{\ss}nahmen}

\begin{frame}{Prävention \& Detektion}
  \textbf{Technische Ma{\ss}nahmen:}
  
  \begin{columns}[T]
    \begin{column}{0.48\textwidth}
      \begin{block}{Backups}
        \footnotesize 3-2-1-Regel: 3 Kopien, 2 Medien, 1 Offsite
      \end{block}
    \end{column}
    \begin{column}{0.48\textwidth}
      \begin{block}{EDR / Antivirus}
        \footnotesize Verhaltensanalyse in Echtzeit
      \end{block}
    \end{column}
  \end{columns}
  
  \vspace{0.5em}
  
  \begin{exampleblock}{Projekt-Reflektion: Warum Evasion?}
    \footnotesize
    Herkömmliche AV-Systeme scannen oft statisch. Unsere Ransomware umgeht dies durch:
    \begin{itemize}
      \item Dynamisches Nachladen (keine Signatur beim PDF-Download)
      \item Sandbox-Checks (verhindert Cloud-Analyse)
      \item Nutzung von Rust (schwerer zu analysieren)
    \end{itemize}
  \end{exampleblock}
\end{frame}

\begin{frame}{Detektion und Reaktion}
  \begin{columns}[T]
    \begin{column}{0.48\textwidth}
      \textbf{Detektionsansätze:}
      \begin{itemize}
        \item \textbf{Statische Analyse}: Ohne Ausführung
        \item \textbf{Dynamische Analyse}: Sandbox
        \item \textbf{Machine Learning}: Unbekannte Varianten
        \item \textbf{Netzwerk}: Anomalie-Erkennung
      \end{itemize}
    \end{column}
    \begin{column}{0.48\textwidth}
      \textbf{Wichtige Merkmale:}
      \begin{itemize}
        \item API-Aufrufe und Systemverhalten
        \item Datei-/Verzeichnisaktivitäten
        \item Netzwerkverkehrsmuster
        \item Verschlüsselungsoperationen
      \end{itemize}
    \end{column}
  \end{columns}
\end{frame}

\begin{frame}{Incident Response}
  \begin{columns}[T]
    \begin{column}{0.48\textwidth}
      \textbf{Sofortma{\ss}nahmen bei Verdacht:}
      \begin{enumerate}
        \item \alert{Isolation} betroffener Systeme
        \item Identifikation des Ransomware-Stamms
        \item Bewertung der Schadenausbreitung
        \item Benachrichtigung relevanter Stellen
      \end{enumerate}
    \end{column}
    \begin{column}{0.48\textwidth}
      \textbf{Wiederherstellung:}
      \begin{itemize}
        \item Aus Backups (wenn möglich)
        \item Key-Escrow-Mechanismen
        \item Forensische Analyse
        \item Systemhärtung vor Neustart
      \end{itemize}
    \end{column}
  \end{columns}
  
  \vspace{0.5em}
  \begin{center}
    \alert{\textbf{Keine Lösegeldzahlung}} (keine Garantie, finanziert weitere Angriffe)
  \end{center}
\end{frame}

% ============================================================
% SEKTION 7: FORSCHUNG & AUSBLICK
% ============================================================
\section{Forschung \& Ausblick}

\begin{frame}{Aktuelle Forschungslandschaft}
  \begin{columns}[T]
    \begin{column}{0.45\textwidth}
      \textbf{Fokusgebiete:}
      \begin{itemize}
        \item \textbf{72,8\%} der Studien: Detektion
        \item Machine Learning dominiert
        \item Hohe Genauigkeit bei unbekannten Varianten
      \end{itemize}
    \end{column}
    \begin{column}{0.52\textwidth}
      \begin{infobox}{REDFISH: Netzwerkbasierte Früherkennung}
        \begin{itemize}
          \item Monitoring von SMB-Protokoll
          \item Erkennung von Verhaltensmustern
          \item \alert{99\% Erkennung bei <10 Dateien}
        \end{itemize}
      \end{infobox}
    \end{column}
  \end{columns}
  
  \vspace{0.3em}
  \small\textbf{Weitere Ansätze:} Near-Zero-Loss durch Traffic-Wiederherstellung, Moving Target Defense
\end{frame}

\begin{frame}{Zero-Loss Recovery: REDFISH}
  \begin{columns}[T]
    \begin{column}{0.48\textwidth}
      \textbf{Funktionsweise:}
      \begin{itemize}
        \item Analyse des Netzwerk-Traffics
        \item Erkennung anomaler Zugriffsmuster
        \item Aufzeichnung der Dateiinhalte
        \item Automatische Alarmierung
      \end{itemize}
    \end{column}
    \begin{column}{0.48\textwidth}
      \textbf{Vorteile:}
      \begin{itemize}
        \item Früherkennung (99\% bei <10 Dateien)
        \item Wiederherstellung ohne Backups
        \item Minimaler Datenverlust
      \end{itemize}
    \end{column}
  \end{columns}
\end{frame}

% ============================================================
% SEKTION 8: FAZIT
% ============================================================
\section{Fazit}

\begin{frame}{Zusammenfassung}
  \textbf{Takeaways aus der Simulation:}
  \vspace{0.5em}
  
  \begin{columns}[T]
    \begin{column}{0.32\textwidth}
      \begin{block}{Automatisierung}
        \footnotesize
        Angriffe sind hochgradig automatisiert (Delivery Server).
      \end{block}
    \end{column}
    \begin{column}{0.32\textwidth}
      \begin{block}{Social Engineering}
        \footnotesize
        Oft effektiver als technische Exploits.
      \end{block}
    \end{column}
    \begin{column}{0.32\textwidth}
      \begin{block}{Double Extortion}
        \footnotesize
        Nicht nur Verschlüsselung, auch Datendiebstahl.
      \end{block}
    \end{column}
  \end{columns}
  
  \vspace{0.8em}
  
  \begin{center}
    \Large\textbf{Prävention und Awareness sind der beste Schutz.}
  \end{center}
\end{frame}

\begin{frame}{Zukünftige Herausforderungen}
  \begin{columns}[T]
    \begin{column}{0.48\textwidth}
      \textbf{Technologische Entwicklungen:}
      \begin{itemize}
        \item \textbf{KI in Ransomware}: Adaptives Verhalten
        \item \textbf{Quantencomputing}: Bedrohung für Verschlüsselung
        \item \textbf{IoT-Ransomware}: Neue Angriffsflächen
        \item \textbf{Cloud-native}: Angriffe auf Cloud-Infra
      \end{itemize}
    \end{column}
    \begin{column}{0.48\textwidth}
      \textbf{Forschungsbedarf:}
      \begin{itemize}
        \item Echtzeit-Schutz, Zero-Day-Erkennung
        \item Post-Quantum-Kryptographie
        \item Automatisierte Incident Response
        \item Internationale Strafverfolgung
      \end{itemize}
    \end{column}
  \end{columns}
\end{frame}

% ============================================================
% QUELLEN
% ============================================================
\begin{frame}{Quellen}
  \begin{thebibliography}{99}
    \setbeamertemplate{bibliography item}[text]
    \bibitem{hammann} IT-Security Vorlesung, Matthias Hammann, WS24/25.
    \bibitem{age} \textit{The Age of Ransomware: A Survey on the Evolution, Taxonomy, and Research Directions} (Razulla et al.)
    \bibitem{enisa} ENISA (2024): \textit{Threat Landscape Report 2024}
    \bibitem{project} \textit{Edu-Ransomware Repository} (GitHub, 2026) -- Eigene Entwicklung
  \end{thebibliography}
\end{frame}

\end{document}
