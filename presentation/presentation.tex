\documentclass[aspectratio=169, 10pt]{beamer}

% --- Theme & Design ---
\usetheme[progressbar=frametitle, block=fill]{metropolis}

% --- Farben (Modernes Security-Blau Schema) ---
\definecolor{SecBlue}{RGB}{10, 30, 60}       % Sehr dunkles Blau (Background/Text)
\definecolor{TechBlue}{RGB}{0, 85, 150}      % Mittleres Blau (Elemente)
\definecolor{CyberCyan}{RGB}{0, 180, 216}    % Helles Cyan (Highlights)
\definecolor{LightGrey}{RGB}{248, 249, 250}  % Fast Weiß
\definecolor{AlertRed}{RGB}{200, 40, 40}     % Für Bedrohungen

% Farb-Anpassungen des Themes
\setbeamercolor{frametitle}{bg=white, fg=TechBlue}
\setbeamercolor{title separator}{fg=CyberCyan}
\setbeamercolor{progress bar}{fg=CyberCyan, bg=LightGrey}
\setbeamercolor{alerted text}{fg=AlertRed}
\setbeamercolor{normal text}{fg=SecBlue}
\setbeamercolor{block title}{bg=TechBlue, fg=white}
\setbeamercolor{block body}{bg=LightGrey, fg=SecBlue}
\definecolor{mDarkTeal}{HTML}{23373b}
\definecolor{mLightBrown}{HTML}{EB811B}
\definecolor{mGray}{HTML}{7f8c8d}
\definecolor{mLightGray}{HTML}{ecf0f1}
\definecolor{mRed}{HTML}{e74c3c}
\definecolor{mBlue}{HTML}{3498db}
\definecolor{mGreen}{HTML}{27ae60}
% --- Pakete ---
\usepackage[utf8]{inputenc}
\usepackage[ngerman]{babel}
\usepackage[utf8]{inputenc}
\usepackage[T1]{fontenc}
\usepackage{tikz}
\usetikzlibrary{shapes, arrows, positioning, shadows, calc, fit, shapes.geometric, patterns, matrix}
\usetikzlibrary{shapes, shapes.symbols, arrows, positioning, shadows, calc, fit, patterns, matrix}
\usepackage{booktabs}
\usepackage{pifont} % Für Checkmarks
\usepackage{listings}

% --- Code Listing Style ---
\lstset{
    basicstyle=\ttfamily\scriptsize\color{SecBlue},
    keywordstyle=\color{TechBlue}\bfseries,
    commentstyle=\color{gray!80},
    stringstyle=\color{teal},
    numberstyle=\tiny\color{gray},
    breaklines=true,
    frame=l,
    framesep=5pt,
    framerule=2pt,
    rulecolor=\color{CyberCyan},
    backgroundcolor=\color{LightGrey},
    showstringspaces=false,
    language=bash
}

% --- Meta Data ---
\title{Ransomware: Analyse \& Simulation}
\subtitle{Bedrohungslage, Kill-Chain und technische Implementierung (Rust/Python)}
\author{\textbf{Demo Präsentation}}
\date{\today}

\begin{document}
\shorthandoff{"} % Wichtig für Babel/TikZ Kompatibilität

% 1. Title Page
\maketitle

% ==============================================================================
% TEIL 1: THEORIE & BEDROHUNGSLAGE (ERWEITERT)
% ==============================================================================

\section{Teil I: Das Ökosystem Ransomware}

% --- Folie 1: Definition & Evolution ---
\begin{frame}{Was ist Ransomware?}
    \begin{columns}[T,onlytextwidth]
        \column{0.6\textwidth}
            \textbf{Definition:}
            Malware, die den Zugriff auf Daten oder Systeme durch Verschlüsselung verhindert und für die Freigabe ein Lösegeld fordert.
            
            \vspace{1em}
            \textbf{Die Evolution der Erpressung:}
            \begin{itemize}
                \item[\textbf{Gen 1:}] \textbf{Single Extortion:} Nur Verschlüsselung.
                \item[\textbf{Gen 2:}] \textbf{Double Extortion:} + Diebstahl sensibler Daten (Drohung mit Leak-Seiten).
                \item[\textbf{Gen 3:}] \textbf{Triple Extortion:} + DDoS-Attacken oder Belästigung von Kunden/Partnern des Opfers.
            \end{itemize}

        \column{0.4\textwidth}
            \centering
            % TikZ: Die Eskalationsstufen
            \begin{tikzpicture}[node distance=0.5cm]
                \node[draw=TechBlue, fill=LightGrey, rounded corners, minimum width=2.5cm] (L1) {Encryption};
                \node[draw=TechBlue, fill=CyberCyan!30, rounded corners, minimum width=2.5cm, below=of L1] (L2) {Data Leak};
                \node[draw=AlertRed, fill=AlertRed!10, rounded corners, minimum width=2.5cm, below=of L2] (L3) {DDoS / Harassment};
                
                \draw[->, thick, color=gray] (L1) -- (L2);
                \draw[->, thick, color=gray] (L2) -- (L3);
            \end{tikzpicture}
    \end{columns}
\end{frame}

% --- Folie 2: RaaS Geschäftsmodell ---
\begin{frame}{Das Geschäftsmodell: Ransomware-as-a-Service (RaaS)}
    Ransomware ist heute eine hochprofessionelle Industrie mit Arbeitsteilung.
    
    \vspace{1em}
    \centering
    \begin{tikzpicture}[
        node distance=1.5cm,
        actor/.style={rectangle, draw=SecBlue, thick, fill=white, rounded corners, minimum height=1.2cm, minimum width=3cm, align=center, drop shadow},
        arrow/.style={->, >=stealth, thick, color=TechBlue}
    ]
        % Actors
        \node[actor] (dev) {\textbf{Core Developers}\\(Operator)};
        \node[actor, right=3cm of dev] (aff) {\textbf{Affiliates}\\(Angreifer)};
        \node[actor, below=1.5cm of aff, draw=AlertRed] (victim) {\textbf{Opfer}\\(Unternehmen)};
        
        % Relations
        \draw[arrow] ([yshift=5pt]dev.east) -- node[above, font=\tiny] {Stellt Malware \& C2} ([yshift=5pt]aff.west);
        \draw[arrow] ([yshift=-5pt]aff.west) -- node[below, font=\tiny] {30\% Provision} ([yshift=-5pt]dev.east);
        
        \draw[arrow, color=AlertRed] (aff) -- node[right, font=\tiny] {Infektion} (victim);
        \draw[arrow, dashed] (victim) -- node[left, font=\tiny, align=right] {Lösegeld\\(Krypto)} ($(dev.south)!0.5!(aff.south)$);
        
    \end{tikzpicture}
    
    \vspace{1em}
    \footnotesize
    \textbf{Vorteil für Kriminelle:} Die Entwickler müssen sich nicht die Hände schmutzig machen (Einbruch), die Affiliates müssen nicht programmieren können.
\end{frame}

% --- Folie 3: Initial Access ---
\begin{frame}{Wie kommen sie rein? (Initial Access Vectors)}
    Die häufigsten Einfallstore für Ransomware-Affiliates:
    
    \vspace{1.5em}
    \centering
    \begin{tikzpicture}
        % Achsen
        \draw[gray!50] (0,0) -- (9,0);
        \draw[gray!50] (0,0) -- (0,3.5);
        
        % Balken 1: Phishing
        \draw[fill=TechBlue] (0, 2.8) rectangle (6.5, 3.3);
        \node[right, font=\footnotesize] at (6.5, 3.05) {Phishing / Social Eng. ($\sim$40\%)};
        
        % Balken 2: Vulnerabilities
        \draw[fill=CyberCyan] (0, 1.8) rectangle (4.5, 2.3);
        \node[right, font=\footnotesize] at (4.5, 2.05) {Schwachstellen (VPN/Citrix) ($\sim$25\%)};
        
        % Balken 3: RDP
        \draw[fill=SecBlue] (0, 0.8) rectangle (3.5, 1.3);
        \node[right, font=\footnotesize] at (3.5, 1.05) {Exposed RDP / Credential Stuffing ($\sim$20\%)};
        
        % Balken 4: Supply Chain
        \draw[fill=gray] (0, -0.2) rectangle (1.5, 0.3);
        \node[right, font=\footnotesize] at (1.5, 0.05) {Supply Chain / Sonstiges ($\sim$15\%)};
        
    \end{tikzpicture}
    
    \begin{tikzpicture}[remember picture, overlay]
        \node[anchor=south east, xshift=-10pt, yshift=5pt, font=\tiny\color{gray}] at (current page.south east) {Datenbasis: Aggregierte Reports 2023/2024 (Mandiant, Sophos)};
    \end{tikzpicture}
\end{frame}

% --- Folie 4: Technische Kill Chain ---
\begin{frame}{Die technische "Kill Chain"}
    Vom ersten Zugriff bis zur Verschlüsselung vergehen oft Wochen ("Dwell Time").
    
    \vspace{1.5em}
    \centering
    \begin{tikzpicture}[
        node distance=0.4cm,
        phase/.style={rectangle, draw=TechBlue, thick, fill=white, rounded corners, minimum width=2.6cm, minimum height=1.4cm, align=center, font=\scriptsize, drop shadow},
        arrow/.style={->, >=stealth, thick, color=CyberCyan, line width=1.5pt}
    ]
        % Phasen
        \node[phase] (access) {\textbf{1. Access}\\\tiny Phishing, RDP\\Exploits};
        \node[phase, right=of access] (persist) {\textbf{2. Persistence}\\\& C2\\\tiny Beaconing};
        \node[phase, right=of persist] (lat) {\textbf{3. Recon}\\\& Lateral Mov.\\\tiny AD-Übernahme};
        \node[phase, right=of lat, draw=AlertRed, fill=AlertRed!5] (impact) {\textbf{4. Impact}\\\tiny Data Exfil \&\\Encryption};
        
        % Pfeile
        \draw[arrow] (access) -- (persist);
        \draw[arrow] (persist) -- (lat);
        \draw[arrow] (lat) -- (impact);
        
    \end{tikzpicture}
    
    \vspace{1.5em}
    \begin{itemize}
        \item \textbf{Schritt 2 \& 3} sind entscheidend für die Verteidigung. Sobald Schritt 4 beginnt, ist es meist zu spät.
        \item Unsere Simulation überspringt \textbf{Schritt 3 (Recon \& lateral Movement)}.
    \end{itemize}
\end{frame}

% --- Folie 5: Krypto-Hintergrund ---
\begin{frame}{Exkurs: Warum wir den Key nicht einfach "finden"}
    Professionelle Ransomware nutzt \textbf{hybride Verschlüsselung}, um Geschwindigkeit mit Sicherheit zu kombinieren.
    
    \vspace{1em}
    \centering
    \begin{tikzpicture}[
        node distance=1cm,
        box/.style={rectangle, draw=gray, rounded corners, font=\tiny, align=center, minimum width=1.8cm, minimum height=1cm},
        arrow/.style={->, >=stealth, color=TechBlue}
    ]
        % Symmetrisch
        \node[box, draw=TechBlue] (file) {Datei};
        \node[box, right=of file] (aes) {\textbf{AES Key}\\(Symmetrisch)};
        \node[box, draw=AlertRed, right=of aes] (encfile) {Encrypted\\File};
        
        % Asymmetrisch
        \node[box, above=of aes] (pub) {\textbf{RSA Public Key}\\(Im Code)};
        \node[box, draw=AlertRed, right=of pub] (enckey) {Encrypted\\AES Key};
        
        % Pfeile
        \draw[arrow] (file) -- (aes);
        \draw[arrow] (aes) -- (encfile);
        \draw[arrow] (aes) -- (pub);
        \draw[arrow] (pub) -- (enckey);
        
        \node[right=0.2cm of encfile, font=\tiny, color=gray, align=left] {$\leftarrow$ Schnell,\\aber Key liegt im RAM};
        \node[right=0.2cm of enckey, font=\tiny, color=gray, align=left] {$\leftarrow$ Key ist sicher\\weggeschlossen};

    \end{tikzpicture}
    
    \vspace{1em}
    \raggedright
    \footnotesize
    \textbf{Das Problem:} Der AES-Key (zum Entschlüsseln nötig) wird mit dem Public Key des Angreifers verschlüsselt und aus dem Speicher gelöscht. Nur der Angreifer hat den Private Key zum Wiederherstellen.
    \newline
    \textit{Hinweis: In unserer Demo vereinfachen wir dies und speichern den Key lokal (`rescue.key`).}
\end{frame}

% --- Folie 6: MITRE ATT&CK ---
\begin{frame}{MITRE ATT\&CK Taktiken}
    \centering
    \footnotesize
    \begin{tabular}{p{3.5cm} p{3.5cm} p{3.5cm} p{3.5cm}}
        \toprule
        \textcolor{TechBlue}{\textbf{Initial Access}} & \textcolor{TechBlue}{\textbf{Persistence}} & \textcolor{TechBlue}{\textbf{Command \& Control}} & \textcolor{AlertRed}{\textbf{Impact}} \\
        \midrule
        $\bullet$ Phishing (T1566) & $\bullet$ Reg Run Keys & $\bullet$ Web Protocols & $\bullet$ Data Encrypted \\
        $\bullet$ Exploit Public App & $\bullet$ Create Account & $\bullet$ Non-Standard Port & \quad for Impact \\
        $\bullet$ Valid Accounts & $\bullet$ Scheduled Task & $\bullet$ Encrypted Channel & $\bullet$ Service Stop \\
        \bottomrule
    \end{tabular}

    \vspace{2em}
    \raggedright
    \textbf{Mapping zur Demo:}
    Unsere Rust-Malware nutzt \textit{Registry Run Keys / Systemd} für Persistence und \textit{Non-Standard Ports} (TCP Socket) für C2.
\end{frame}

% --- Folie 5: C2 Context ---
\begin{frame}{Command \& Control (C2) Architektur}
    Die "Lebensader" des Angriffs: C2 steuert Payloads, empfängt Keys und exfiltrierte Daten.
    
    \vspace{1em}
    \centering
    \begin{tikzpicture}[node distance=1.5cm]
        % Nodes
        \node[draw=SecBlue, fill=LightGrey, rounded corners, align=center] (corp) {
            \textbf{Corporate Network}\\
            \scriptsize Client A \quad Client B
        };
        
        \node[draw=AlertRed, thick, fill=white, right=2cm of corp, align=center, rounded corners] (infected) {Infected\\Host};
        
        % KORREKTUR: Cloud als Node-Shape, nicht als Befehl
        \node[cloud, cloud puffs=10, draw=gray, aspect=2, right=2cm of infected, align=center] (internet) {Internet};
        
        \node[draw=TechBlue, fill=SecBlue, text=white, rounded corners, right=0.5cm of internet] (c2) {C2 Server};
        
        % Connections
        \draw[thick] (corp) -- (infected);
        \draw[<->, dashed, thick, color=AlertRed] (infected) -- node[above, font=\tiny] {HTTPS Beaconing} node[below, font=\tiny] {DNS Tunneling} (internet);
        \draw[<->, thick] (internet) -- (c2);
        
    \end{tikzpicture}
    
    \vspace{1em}
    \raggedright
    \small
    Nutzung unauffälliger Kanäle (Port 443, DNS), um in der Masse des Traffics unterzugehen.

    \begin{tikzpicture}[remember picture, overlay]
        \node[anchor=south east, xshift=-10pt, yshift=5pt, font=\tiny\color{gray}] at (current page.south east) {Quellen: Halcyon – C2 in Ransomware; Packetlabs Deep Dive};
    \end{tikzpicture}
\end{frame}


% --- Folie 6: Defense ---
\begin{frame}{Erkennung \& Abwehr (Defense in Depth)}
    
    \begin{columns}
        \column{0.6\textwidth}
            \textbf{Technische Maßnahmen:}
            \begin{itemize}
                \item \textbf{C2-Detection:} Blockieren von bekannten Malicious IPs/Domains \& Analyse von Beaconing-Mustern.
                \item \textbf{EDR (Endpoint Detection):} Überwachung auf Prozess-Injections und Massen-Dateiänderungen.
            \end{itemize}
            
            \vspace{0.5em}
            \textbf{Organisatorisch:}
            \begin{itemize}
                \item Offline-Backups (Immutable).
                \item Netzwerksegmentierung.
            \end{itemize}

        \column{0.4\textwidth}
            \centering
            % TikZ Layers
            \begin{tikzpicture}
                \node[draw=TechBlue, fill=white, minimum width=3cm, rounded corners] (L1) {Awareness};
                \node[draw=TechBlue, fill=LightGrey, minimum width=3cm, rounded corners, above=0.2cm of L1] (L2) {Firewall / IDS};
                \node[draw=TechBlue, fill=CyberCyan!30, minimum width=3cm, rounded corners, above=0.2cm of L2] (L3) {EDR / AV};
                \node[draw=SecBlue, fill=TechBlue, text=white, minimum width=3cm, rounded corners, above=0.2cm of L3] (L4) {Backup};
                
                \node[right=0.2cm of L4, font=\tiny, color=gray] {Letzte Linie};
            \end{tikzpicture}
    \end{columns}

    \begin{tikzpicture}[remember picture, overlay]
        \node[anchor=south east, xshift=-10pt, yshift=5pt, font=\tiny\color{gray}] at (current page.south east) {Quellen: BSI; Darktrace – Early Signs of Ransomware};
    \end{tikzpicture}
\end{frame}


% ==============================================================================
% TEIL 2: TECHNISCHE SIMULATION (Praxis)
% ==============================================================================

\section{Teil II: Technische Simulation}

\begin{frame}{Architektur der Simulation}
    \centering
    \begin{tikzpicture}[
        node distance=2cm,
        box/.style={rectangle, draw=none, fill=LightGrey, rounded corners, minimum height=1.5cm, minimum width=3.5cm, align=center, drop shadow},
        label/.style={font=\tiny, color=gray},
        arrow/.style={->, >=stealth, thick, color=TechBlue}
    ]
        % Nodes
        \node[box] (victim) {\textbf{\textcolor{CyberCyan}{RUST AGENT}}\\(Victim Client)};
        \node[box, right=4cm of victim] (c2) {\textbf{\textcolor{TechBlue}{PYTHON C2}}\\(Attacker Server)};
        
        % Firewall Wall Visualisierung
        \draw[line width=2pt, dashed, color=gray!50] (4, -1.8) -- (4, 1.8);
        \node[fill=white, text=gray, font=\tiny, inner sep=2pt] at (4,0) {Firewall / NAT};
        
        % Arrows
        \draw[arrow] ([yshift=8pt]victim.east) -- node[above, font=\tiny, align=center] {1. Reverse TCP Connect\\(Port 4444)} ([yshift=8pt]c2.west);
        \draw[arrow, dashed] ([yshift=-8pt]c2.west) -- node[below, font=\tiny, align=center] {2. Commands\\(encrypt/decrypt)} ([yshift=-8pt]victim.east);
        
    \end{tikzpicture}

    \vspace{1.5em}
    \begin{itemize}
        \item \textbf{Reverse Shell Prinzip:} Der Agent baut die Verbindung auf (Outbound), um Inbound-Firewall-Regeln zu umgehen.
        \item \textbf{Technologie:} Rust (Client) für Performance und Sicherheit, Python (Server) für einfache Handhabung.
    \end{itemize}
\end{frame}

\begin{frame}{Der Agent: Warum Rust?}
    \begin{columns}[T,onlytextwidth]
        \column{0.5\textwidth}
            \begin{block}{Technische Vorteile}
                \begin{itemize}
                    \item \textbf{Memory Safety:} Keine Buffer Overflows (weniger Abstürze beim Opfer).
                    \item \textbf{Zero Runtime:} Keine Installation von Python/Java nötig (Single Binary).
                    \item \textbf{Cross-Platform:} Ein Code für Linux, Windows \& macOS.
                \end{itemize}
            \end{block}
        
        \column{0.45\textwidth}
            \begin{block}{Evasion (Erkennung)}
                \begin{itemize}
                    \item Komplexer Maschinencode erschwert Reverse Engineering.
                    \item Geringere Erkennungsrate bei klassischen AV-Signaturen im Vergleich zu C/C++.
                \end{itemize}
            \end{block}
    \end{columns}
\end{frame}

% Cryptography Flow
\begin{frame}{Kryptographie: Der "Atomic Lock"}
    Ein kritischer Fehler bei Malware ist Datenverlust durch Abstürze. Unser Agent nutzt ein atomares Verfahren:

    \vspace{1.5em}
    \centering
    \vspace{-0.5cm}
    \begin{tikzpicture}[
        node distance=0.5cm,
        phase/.style={rectangle, rounded corners, text width=2cm, align=center, minimum height=1.6cm, draw=mDarkTeal, thick, fill=white, drop shadow},
        arrow/.style={->, >=stealth, thick, color=mLightBrown}
    ]
        % Phase 1
        \node[phase, fill=mRed!20] (p1) {
            \textbf{\small 1. Initial}\\
            \textbf{\small Access}\\[0.1cm]
            \tiny Phishing\\RDP\\Exploits
        };
        
        % Phase 2
        \node[phase, fill=mLightBrown!20, right=of p1] (p2) {
            \textbf{\small 2. Execution}\\[0.1cm]
            \tiny Payload\\Download
        };
        
        % Phase 3
        \node[phase, fill=mBlue!20, right=of p2] (p3) {
            \textbf{\small 3. Persistence}\\[0.1cm]
            \tiny Lateral\\Movement
        };
        
        % Phase 4
        \node[phase, fill=mGray!30, right=of p3] (p4) {
            \textbf{\small 4. Exfiltration}\\[0.1cm]
            \tiny Datenklau
        };
        
        % Phase 5
        \node[phase, fill=mRed!40, right=of p4] (p5) {
            \textbf{\small 5. Impact}\\[0.1cm]
            \tiny Verschlüsselung
        };
        
        \draw[arrow] (p1) -- (p2);
        \draw[arrow] (p2) -- (p3);
        \draw[arrow] (p3) -- (p4);
        \draw[arrow] (p4) -- (p5);
        
        % Zeitangabe unten
        \node[below=0.4cm of p3, font=\tiny, color=mGray, text width=10cm, align=center] {
            Typische Dwell Time: 24-48 Stunden bis zur Verschlüsselung
        };
    \end{tikzpicture}
\end{frame}

% =================================================================
\section{Angriffsvektoren und Ziele}
% =================================================================

\begin{frame}{Vektoren und Betroffene Sektoren}
    \begin{columns}[T]
        
        % Linke Spalte: Vektoren
        \column{0.48\textwidth}
            \textbf{Häufigste Vektoren} \citeg{1, 4}
            \begin{itemize}
                \setlength\itemsep{0.5em}
                \item \textbf{Phishing:} Dominanter Vektor (Mails \& Links)
                \item \textbf{RDP:} Brute-Force auf Remote Desktop Protokolle
                \item \textbf{Exploits:} Ausnutzung ungepatchter Software-Lücken
                \item \textbf{Supply Chain:} Kompromittierung von Software-Updates
            \end{itemize}
            
            \vspace{0.5cm}
            
            \textbf{Aktuelle Trends}
            \begin{itemize}
                \setlength\itemsep{0.5em}
                \item \textbf{KI-Einsatz:} Nutzung für Phishing \& Evasion
                \item \textbf{Rootkits:} Techniken zur Verschleierung
            \end{itemize}

        % Rechte Spalte: Sektoren
        \column{0.48\textwidth}
            \centering
            \textbf{Top Ziele nach Sektoren \citeg{4}}
            \vspace{0.2cm}
            
            \begin{tikzpicture}[
                node distance=0.2cm,
                boxstyle/.style={
                    rectangle, 
                    rounded corners, 
                    minimum height=0.8cm, 
                    text width=5.5cm,
                    align=left, 
                    inner sep=6pt,
                    anchor=north
                }
            ]
                % 1. Industrie
                \node[boxstyle, fill=mDarkTeal, text=white] (ind) {
                    \textbf{1. Industrie \& Fertigung}\\
                    \footnotesize{Am häufigsten angegriffen (21,78\%)}
                };

                % 2. Transport
                \node[boxstyle, fill=mDarkTeal!85, text=white, below=of ind] (trans) {
                    \textbf{2. Transportsektor}\\
                    \footnotesize{Zweithäufigstes Ziel (21,05\%)}
                };

                % 3. Kritische Infrastruktur
                \node[boxstyle, fill=mDarkTeal!70, text=white, below=of trans] (kritis) {
                    \textbf{3. Kritische Infrastrukturen}\\
                    \footnotesize{Fokus auf Transport \& \textbf{Energie}}
                };

                % 4. Weitere Sektoren
                \node[boxstyle, draw=mDarkTeal, fill=mLightGray!20, text=mDarkTeal, below=of kritis] (others) {
                    \textbf{Weitere Hauptziele:}\\
                    \footnotesize{
                    $\bullet$ Öffentliche Verwaltung\\
                    $\bullet$ Gesundheitswesen\\
                    $\bullet$ Bankwesen \& Finanzen
                    }
                };
            \end{tikzpicture}
    \end{columns}
\end{frame}

% --- Demo Workflow ---
\begin{frame}{Ablauf der Demonstration}
    Wie die Komponenten im Test zusammenspielen:
    
    \vspace{1em}
    
    \begin{enumerate}
        \item \textbf{Setup:} Starten des Python C2 Servers (Listener auf Port 4444).
        \item \textbf{Infektion:} Ausführen der \texttt{rust-mw} Binary auf dem Zielsystem.
        \item \textbf{Handshake:} Client meldet sich beim Server ("New Session").
        \item \textbf{Angriff:} 
            \begin{itemize}
                \item API-Aufrufe
                \item Datei-/Verzeichnisaktivität
                \item Netzwerkverkehr
                \item Entropie-Analyse
            \end{itemize}

        % Spalte 2: Analyse-Methoden
        \column{0.55\textwidth}
            \centering
            \textbf{Analyseansätze} \citeg{1, 4}
            \vspace{0.2cm}
            \begin{tikzpicture}[
                box/.style={rectangle, draw=none, rounded corners, text width=6cm, inner sep=8pt, align=left, drop shadow}
            ]
                % Statisch
                \node[box, fill=mDarkTeal!15] (static) {
                    \textbf{1. Statische Analyse}\\
                    \footnotesize Untersuchung der Binärdatei \textit{ohne} Ausführung (Code-Struktur, Signaturen)
                };
                
                % Dynamisch
                \node[box, fill=mLightBrown!20, below=0.3cm of static] (dynamic) {
                    \textbf{2. Dynamische Analyse}\\
                    \footnotesize Beobachtung des Verhaltens in isolierter Umgebung (\textbf{Sandbox})
                };
                
                % Hybrid
                \node[box, fill=mBlue!15, below=0.3cm of dynamic] (hybrid) {
                    \textbf{3. Hybrid-Ansatz}\\
                    \footnotesize Kombination beider Methoden für optimale Erkennung
                };
            \end{tikzpicture}
    \end{columns}
\end{frame}

\begin{frame}{Präventions- und Minderungsstrategien}
    \footnotesize

    \begin{columns}[T,onlytextwidth]

        \column{0.32\textwidth}
        \setbeamercolor{block title}{bg=mDarkTeal,fg=white}
        \setbeamercolor{block body}{bg=mDarkTeal!90,fg=white}
        \setlength{\parskip}{0.4em}
        \begin{block}{Echtzeit-Schutz}
            \vspace{0.3em}
            \textbf{Ziel:} Identifizierung von Zero-Day-Ransomware

            Systeme zur Erkennung und Unterbrechung von Angriffen
            während der Ausführung
            \vspace{0.3em}
        \end{block}

        \column{0.32\textwidth}
        \setbeamercolor{block title}{bg=mLightBrown,fg=white}
        \setbeamercolor{block body}{bg=mLightBrown!90,fg=white}
        \setlength{\parskip}{0.4em}
        \begin{block}{Abwehr}
            \vspace{0.3em}
            \textbf{Policies:} Strikte Dateizugriffsrichtlinien

            \textbf{MTD:} \textit{Moving Target Defense}
            (z.\,B. rotierende Dateiendungen)
            \vspace{0.3em}
        \end{block}

        \column{0.32\textwidth}
        \setbeamercolor{block title}{bg=gray,fg=white}
        \setbeamercolor{block body}{bg=gray!85,fg=white}
        \setlength{\parskip}{0.4em}
        \begin{block}{Recovery}
            \vspace{0.3em}
            \textbf{Backups:} Cloud-Lösungen

            \textbf{Hardware:} SSD Flash-Level Recovery

            \textbf{Krypto:} Key-Escrow
            \vspace{0.3em}
        \end{block}

    \end{columns}

    \vspace{0.3em}
    {\tiny Basierend auf Strategien aus \textit{Age of Ransomware} \citeg{1}}

\end{frame}

% =================================================================
\section{REDFISH: Netzwerkbasierte Detektion}
% =================================================================
\begin{frame}{Netzwerkbasierte Detektion: REDFISH}
      \vspace{0.2cm}
    \begin{columns}
        \column{0.4\textwidth}
            \textbf{Konzept} \citeg{2}
            \vspace{0.1cm}
            \begin{itemize}
                \item Analyse von \textbf{SMB-Traffic} zu geteilten Laufwerken (NAS)
                \item Keine Software auf Endgeräten nötig
                \item Monitoring von \textbf{Verhaltensmustern}
                \item Schnelles Lesen/Schreiben/Löschen
            \end{itemize}
            
            \vspace{0.2cm}
            \textbf{Vorteile:}
            \begin{itemize}
                \item Zentrale Überwachung
                \item Keine Client-Installation
                \item Früherkennung möglich
            \end{itemize}

        \column{0.6\textwidth}
            \centering
            \begin{tikzpicture}[
                scale=0.85, transform shape,
                flowbox/.style={rectangle, draw=mDarkTeal, thick, fill=white, rounded corners, text width=2.5cm, align=center, minimum height=1.2cm, drop shadow},
                arrow/.style={->, >=stealth, thick, color=mLightBrown}
            ]
                % Ablauf
                \node[flowbox] (monitor) {Traffic Analyse (SMB)};
                \node[flowbox, right=0.5cm of monitor] (detect) {Anomalie-Erkennung};
                \node[flowbox, right=0.5cm of detect, fill=mLightBrown!20] (action) {\textbf{Blockade} < 20 Sek.};
                
                \draw[arrow] (monitor) -- (detect);
                \draw[arrow] (detect) -- (action);
                
                % Ergebnis-Box darunter
                \node[rectangle, fill=mDarkTeal!10, draw=none, rounded corners, below=0.5cm of detect, text width=7cm, align=center, inner sep=8pt] {
                    \textbf{Ergebnis:} 99\% Erkennung bevor 10 Dateien verloren gehen\\
                    \vspace{0.1cm}
                    \footnotesize \textit{„Near-0-Loss" Szenario: Wiederherstellung aus Netzwerk-Paketen möglich}
                };
            \end{tikzpicture}
    \end{columns}
\end{frame}

\begin{frame}{REDFISH: Architektur und Funktionsweise}
    \centering
    \begin{tikzpicture}[
        node distance=1cm,
        component/.style={rectangle, rounded corners, minimum width=2.5cm, minimum height=1cm, align=center, draw=mDarkTeal, thick, fill=white, drop shadow},
        arrow/.style={->, >=stealth, thick}
    ]
        % Endgeräte
        \node[component, fill=mLightGray!30] (client1) {Client 1};
        \node[component, fill=mLightGray!30, right=0.3cm of client1] (client2) {Client 2};
        \node[component, fill=mLightGray!30, right=0.3cm of client2] (client3) {Client 3};
        
        % Netzwerk
        \node[component, fill=mBlue!20, below=1.5cm of client2, minimum width=8cm] (network) {
            \textbf{Netzwerk-Switch}\\
            \footnotesize Mirror Port für Traffic-Analyse
        };
        
        % REDFISH System
        \node[component, fill=mLightBrown!30, below=1.5cm of network, minimum width=5cm, minimum height=1.5cm] (redfish) {
            \textbf{REDFISH System}\\
            \footnotesize 
            $\bullet$ Pattern Recognition\\
            $\bullet$ Anomaly Detection\\
            $\bullet$ Alert Generation
        };
        
        % NAS/Storage
        \node[component, fill=mGreen!20, right=2cm of network] (nas) {
            \textbf{NAS}\\
            \footnotesize Shared\\Storage
        };
        
        % Pfeile
        \draw[arrow, mGray] (client1) -- (network);
        \draw[arrow, mGray] (client2) -- (network);
        \draw[arrow, mGray] (client3) -- (network);
        \draw[arrow, mGray] (network) -- (nas);
        \draw[arrow, mLightBrown, thick] (network) -- (redfish) node[midway, right, font=\tiny] {Mirror};
        
        % Alarm
        \node[component, fill=mRed!30, left=2cm of redfish, minimum width=2cm] (alarm) {
            \textbf{Alert}\\
            \footnotesize Admin
        };
        \draw[arrow, mRed, very thick] (redfish) -- (alarm);
    \end{tikzpicture}
\end{frame}

\begin{frame}{REDFISH: Erkennungsmetriken}
    \vspace{0.2cm}
    \begin{columns}
        \column{0.5\textwidth}
            \textbf{Überwachte Parameter:}
            \begin{itemize}
                \item \textbf{Read Rate:} Datei-Lesezugriffe pro Zeiteinheit
                \item \textbf{Write Rate:} Schreibvorgänge (verschlüsselte Daten)
                \item \textbf{Delete Rate:} Löschung von Originaldateien
                \item \textbf{Entropy:} Randomness der geschriebenen Daten
            \end{itemize}
            
            \vspace{0.3cm}
            \textbf{Schwellwerte:}
            \begin{itemize}
                \item Anpassbar pro Netzwerk
                \item Machine Learning für Baseline
                \item False Positive Rate: < 1\%
            \end{itemize}

        \column{0.5\textwidth}
            \begin{tikzpicture}
                \begin{axis}[
                    width=\textwidth,
                    height=6cm,
                    xlabel={Zeit (Sekunden)},
                    ylabel={Anzahl Operationen},
                    legend pos=north west,
                    ymajorgrids=true,
                    grid style=dashed,
                ]
                    % Normal
                    \addplot[color=mBlue, thick] coordinates {
                        (0,2) (5,3) (10,2) (15,4) (20,3)
                    };
                    \addlegendentry{Ransomware}
                    
                    % Alarm Threshold
                    \addplot[color=mLightBrown, thick, dotted] coordinates {
                        (0,20) (20,20)
                    };
                    \addlegendentry{Alarm-Schwelle}
                \end{axis}
            \end{tikzpicture}
    \end{columns}
\end{frame}

% =================================================================
\section{Zero-Loss Recovery}
% =================================================================

\begin{frame}{Zero-Loss Recovery Konzept}
    \textbf{Grundidee:} \citeg{2}
    \begin{itemize}
        \item Wiederherstellung von Dateien \textbf{bevor} sie dauerhaft verloren gehen
        \item Nutzung von Netzwerk-Traffic für Rekonstruktion
        \item Minimierung des Datenverlusts auf < 10 Dateien
    \end{itemize}
    
    \vspace{0.5cm}
    
    \begin{center}
    \begin{tikzpicture}[
        node distance=0.8cm,
        box/.style={rectangle, rounded corners, minimum width=3cm, minimum height=1cm, align=center, draw=mDarkTeal, thick, drop shadow}
    ]
        % Prozess
        \node[box, fill=mLightGray!30] (detect) {1. Detektion\\< 20 Sek};
        \node[box, fill=mLightBrown!20, right=of detect] (block) {2. Blockade\\Sofortig};
        \node[box, fill=mBlue!20, right=of block] (capture) {3. Packet\\Capture};
        \node[box, fill=mGreen!20, right=of capture] (restore) {4. Restore\\aus Packets};
        
        \draw[->, >=stealth, very thick, mDarkTeal] (detect) -- (block);
        \draw[->, >=stealth, very thick, mDarkTeal] (block) -- (capture);
        \draw[->, >=stealth, very thick, mDarkTeal] (capture) -- (restore);
    \end{tikzpicture}
    \end{center}
    
    \vspace{0.5cm}
    
    \begin{alertblock}{Kernvorteil}
        Im Gegensatz zu traditionellen Backups: \textbf{Echtzeitwiederherstellung} ohne Zeitverzögerung oder Datenverlust
    \end{alertblock}
\end{frame}

\begin{frame}{Vergleich: Traditionelle vs. Zero-Loss Recovery}
    \begin{center}
    \begin{tikzpicture}
        \begin{axis}[
            ybar,
            bar width=25pt,
            width=0.9\textwidth,
            height=5.5cm,
            symbolic x coords={Traditionelles Backup, REDFISH Zero-Loss},
            xtick=data,
            ylabel={Verlorene Dateien},
            ymin=0,
            ymax=1200,
            nodes near coords,
            nodes near coords align={vertical},
            legend pos=north west,
            ymajorgrids=true,
            grid style=dashed,
            enlarge x limits=0.3
        ]
            \addplot[fill=mRed!60] coordinates {
                (Traditionelles Backup, 1000)
                (REDFISH Zero-Loss, 8)
            };
            \addlegendentry{Durchschnittlicher Verlust}
        \end{axis}
    \end{tikzpicture}
    \end{center}
    
    \vspace{0.1cm}
    
    \begin{columns}[t]
        \column{0.48\textwidth}
            \textbf{Traditionell:}
            \begin{itemize}
                \item Backup-Intervall: Stunden/Tage
                \item Wiederherstellung: Manuell
                \item Datenverlust: Signifikant
            \end{itemize}
        
        \column{0.48\textwidth}
            \textbf{REDFISH:}
            \begin{itemize}
                \item Detektion: < 20 Sekunden
                \item Wiederherstellung: Automatisch
                \item Datenverlust: < 10 Dateien
            \end{itemize}
    \end{columns}
\end{frame}

% =================================================================
\section{Praktische Implementierung}
% =================================================================

\begin{frame}{Deployment-Szenarien}
    \begin{columns}
        \column{0.5\textwidth}
            \textbf{Enterprise-Umgebung:}
            \begin{itemize}
                \item Integration in bestehende SIEM-Systeme
                \item Zentrale Überwachung aller Standorte
                \item Skalierbar für große Netzwerke
            \end{itemize}
            
            \vspace{0.5cm}
            
            \textbf{Anforderungen:}
            \begin{itemize}
                \item Mirror Port am Core Switch
                \item Dedizierte Hardware für Analyse
                \item Netzwerk-Bandbreite für Packet Capture
            \end{itemize}

        \column{0.5\textwidth}
            \centering
            \begin{tikzpicture}[scale=0.8, transform shape]
                % Layers
                \node[rectangle, draw=mDarkTeal, thick, fill=mLightGray!20, minimum width=6cm, minimum height=1cm] (siem) at (0,3) {SIEM Integration};
                
                \node[rectangle, draw=mDarkTeal, thick, fill=mLightBrown!20, minimum width=6cm, minimum height=1cm] (redfish) at (0,1.5) {REDFISH Engine};
                
                \node[rectangle, draw=mDarkTeal, thick, fill=mBlue!20, minimum width=6cm, minimum height=1cm] (network) at (0,0) {Network Layer};
                
                \node[rectangle, draw=mDarkTeal, thick, fill=mGreen!20, minimum width=6cm, minimum height=1cm] (storage) at (0,-1.5) {Storage/NAS};
                
                % Connections
                \draw[->, >=stealth, thick] (storage) -- (network);
                \draw[->, >=stealth, thick] (network) -- (redfish);
                \draw[->, >=stealth, thick] (redfish) -- (siem);
            \end{tikzpicture}
    \end{columns}
\end{frame}

\begin{frame}{Performance und Evaluierung}
    \textbf{Testumgebung:} \citeg{2}
    \begin{itemize}
        \item 1000 Dateien verschiedener Typen
        \item 10 verschiedene Ransomware-Varianten
        \item Reale Netzwerk-Bedingungen
    \end{itemize}
    
    \vspace{0.5cm}
    
    \begin{center}
    \begin{tikzpicture}
        \begin{axis}[
            width=0.7\textwidth,
            height=5cm,
            xlabel={Ransomware Variante},
            ylabel={Erkennungsrate (\%)},
            ymin=95, ymax=100,
            xtick={1,2,3,4,5,6,7,8,9,10},
            ymajorgrids=true,
            grid style=dashed,
            legend pos=south east
        ]
            \addplot[color=mGreen, mark=*, thick] coordinates {
                (1,99.2) (2,99.5) (3,98.9) (4,99.8) (5,99.1)
                (6,99.4) (7,98.7) (8,99.6) (9,99.3) (10,99.0)
            };
            \addlegendentry{REDFISH}
        \end{axis}
    \end{tikzpicture}
    \end{center}
    
    \textbf{Durchschnittliche Erkennungsrate: 99.25\%}
\end{frame}

% =================================================================
\section{Zukunftsperspektiven}
% =================================================================

\begin{frame}{Herausforderungen und offene Fragen}
    \begin{columns}[t]
        \column{0.48\textwidth}
            \textbf{Technische Herausforderungen:}
            \begin{itemize}
                \item \textbf{Verschlüsselter Traffic:} TLS/SSL erschwert Analyse
                \item \textbf{Skalierung:} Große Netzwerke mit hohem Throughput
                \item \textbf{Performance:} Echtzeit-Analyse bei hoher Last
                \item \textbf{False Positives:} Legitime Massenoperationen
            \end{itemize}

        \column{0.48\textwidth}
            \textbf{Forschungsrichtungen:}
            \begin{itemize}
                \item \textbf{KI-Integration:} Verbesserte Mustererkennung
                \item \textbf{Cloud-Adaption:} Schutz für Cloud-Storage
                \item \textbf{IoT-Schutz:} Erweiterung auf IoT-Netzwerke
                \item \textbf{Automatisierung:} Selbstlernende Systeme
            \end{itemize}
    \end{columns}
    
    \vspace{0.8cm}
    
    \begin{alertblock}{Wichtig}
        Ransomware entwickelt sich ständig weiter - Detektionssysteme müssen adaptiv sein
    \end{alertblock}
\end{frame}

\begin{frame}{Ausblick: Die Zukunft der Ransomware-Abwehr}
    \centering
    \begin{tikzpicture}[
        node distance=0.4cm,
        future/.style={rectangle, rounded corners, text width=4cm, minimum height=2.5cm, align=center, draw=mDarkTeal, thick, fill=white, drop shadow}
    ]
        \node[future, fill=mBlue!20] (ai) {
            \textbf{KI \& ML}\\
            \vspace{0.2cm}
            \footnotesize
            Adaptive Erkennung\\
            Predictive Analytics\\
            Automated Response
        };
        
        \node[future, fill=mGreen!20, right=of ai] (zero) {
            \textbf{Zero Trust}\\
            \vspace{0.2cm}
            \footnotesize
            Micro-Segmentation\\
            Least Privilege\\
            Continuous Verification
        };
        
        \node[future, fill=mLightBrown!20, right=of zero] (quantum) {
            \textbf{Quantum-Ready}\\
            \vspace{0.2cm}
            \footnotesize
            Post-Quantum Crypto\\
            Quantum Detection\\
            Future-Proof Security
        };
    \end{tikzpicture}
    
    \vspace{0.8cm}
    
    \textbf{Kernbotschaft:}
    \begin{itemize}
        \item Defense in Depth bleibt essentiell
        \item Kombination aus Prävention, Detektion und Recovery
        \item Kontinuierliche Anpassung an neue Bedrohungen
    \end{itemize}
\end{frame}

% =================================================================
\section{Zusammenfassung}
% =================================================================

\begin{frame}{Key Takeaways}
    \begin{enumerate}
        \item \textbf{Bedrohungslage:} Ransomware ist eine der größten Cyber-Bedrohungen mit massiven finanziellen Auswirkungen
        
        \item \textbf{Evolution:} Von einfacher Dateiverschlüsselung zu komplexen Multi-Extortion-Strategien
        
        \item \textbf{Detektion:} ML-basierte Ansätze zeigen hohe Erfolgsraten (> 99\%)
        
        \item \textbf{REDFISH:} Netzwerkbasierte Detektion ermöglicht Near-Zero-Loss Recovery
        
        \item \textbf{Zukunft:} Integration von KI, Zero Trust und Quantum-Readiness notwendig
    \end{enumerate}
    
    \vspace{0.5cm}
    
    \begin{block}{Fazit}
        Ein mehrschichtiger Sicherheitsansatz kombiniert mit modernen Detektionssystemen wie REDFISH bietet den besten Schutz gegen Ransomware
    \end{block}
\end{frame}

% =================================================================
\section*{Quellenverzeichnis}
% =================================================================

\begin{frame}[allowframebreaks]{Referenzen}
    \setbeamertemplate{bibliography item}[text]
    
    \begin{thebibliography}{9}
        
        \bibitem{1}
        \textbf{Razaulla et al.}
        \newblock \textit{The Age of Ransomware: A Survey on the Evolution, Taxonomy, and Research Directions} (2023)

        \bibitem{2}
        \textbf{Morato et al.}
        \newblock \textit{Ransomware early detection by the analysis of file sharing traffic (REDFISH)}.
        \newblock Journal of Network and Computer Applications, 2018.

        \bibitem{3}
        \textbf{Oz et al.}
        \newblock \textit{A Survey on Ransomware: Evolution, Taxonomy, and Defense Solutions}.
        \newblock ACM Computing Surveys, 2022.

        \bibitem{4}
        \textbf{ENISA}
        \newblock \textit{ENISA Threat Landscape 2024}.
        \newblock European Union Agency for Cybersecurity, 2024.

    \end{thebibliography}
\end{frame}

\begin{frame}[standout]
    Vielen Dank für Ihre Aufmerksamkeit!
    \vspace{0.5cm}
    
    \normalsize
    \vspace{2em}
    \begin{itemize}
        \item[\checkmark] \textbf{Bedrohung:} Ransomware professionalisiert sich (RaaS, C2).
        \item[\checkmark] \textbf{Technik:} Rust + Sockets ermöglichen potente Malware.
        \item[\checkmark] \textbf{Schutz:} Defense-in-Depth ist alternativlos.
    \end{itemize}
\end{frame}

\end{document}
