\documentclass[aspectratio=169, 10pt]{beamer}

% --- Theme & Design ---
\usetheme[progressbar=frametitle, block=fill]{metropolis}

% --- Farben (Modernes Security-Blau Schema) ---
% Tiefes Blau für Seriosität, leuchtendes Cyan für den "Cyber"-Look
\definecolor{SecBlue}{RGB}{10, 30, 60}       % Sehr dunkles Blau (Hintergrund/Struktur)
\definecolor{TechBlue}{RGB}{0, 85, 150}      % Mittleres Blau (Elemente)
\definecolor{CyberCyan}{RGB}{0, 180, 216}    % Helles Cyan (Highlights/Akzente)
\definecolor{LightGrey}{RGB}{248, 249, 250}  % Fast Weiß

% Farb-Anpassungen des Themes
\setbeamercolor{frametitle}{bg=white, fg=TechBlue}
\setbeamercolor{title separator}{fg=CyberCyan}
\setbeamercolor{progress bar}{fg=CyberCyan, bg=LightGrey}
\setbeamercolor{alerted text}{fg=CyberCyan}
\setbeamercolor{normal text}{fg=SecBlue}
\setbeamercolor{block title}{bg=TechBlue, fg=white}
\setbeamercolor{block body}{bg=LightGrey, fg=SecBlue}

% --- Pakete ---
\usepackage[utf8]{inputenc}
\usepackage[ngerman]{babel}
\usepackage{helvet} 
\usepackage{listings}
\usepackage{tikz}
\usetikzlibrary{shapes, arrows, positioning, shadows, calc, fit}
\usepackage{booktabs}

% --- Code Listing Style (VS Code Dark Theme inspiriert) ---
\lstset{
    basicstyle=\ttfamily\scriptsize\color{SecBlue},
    keywordstyle=\color{TechBlue}\bfseries,
    commentstyle=\color{gray!80},
    stringstyle=\color{teal},
    numberstyle=\tiny\color{gray},
    breaklines=true,
    frame=l, % Linie links
    framesep=5pt,
    framerule=2pt,
    rulecolor=\color{CyberCyan},
    backgroundcolor=\color{LightGrey},
    showstringspaces=false,
    language=bash
}

% --- Meta Data ---
\title{Ransomware Analyse \& Simulation}
\subtitle{Von der Bedrohungslage zur technischen Implementierung in Rust}
\author{\textbf{Demo Präsentation}}
\date{\today}
\titlegraphic{\hfill\footnotesize\ttfamily\color{gray}Security Research v1.0}

\begin{document}
\shorthandoff{"}
% 1. Title Page
\maketitle

% ==============================================================================
% TEIL 1: ALLGEMEINER KONTEXT (Theorie)
% ==============================================================================

\section{Die Bedrohungslage}

\begin{frame}{Ransomware: Status Quo}
    \begin{columns}[T,onlytextwidth]
        \column{0.5\textwidth}
            \textbf{Was ist Ransomware?}
            Schadsoftware, die Daten verschlüsselt und Lösegeld (Krypto-Währung) für die Entschlüsselung fordert.
            
            \vspace{1em}
            \textbf{Aktuelle Trends:}
            \begin{itemize}
                \item \textbf{RaaS:} Ransomware-as-a-Service (Miet-Malware).
                \item \textbf{Double Extortion:} Verschlüsselung + Drohung mit Datenveröffentlichung.
                \item \textbf{Zielgerichtete Angriffe:} Fokus auf Kritische Infrastruktur (KRITIS) \& Industrie.
            \end{itemize}

        \column{0.5\textwidth}
            \centering
            % TikZ Balkendiagramm (Symbolisch für steigende Schäden)
            \begin{tikzpicture}
                \draw[gray!20] (0,0) grid (5,3.5);
                \draw[thick, ->] (0,0) -- (5.2,0) node[right] {\tiny Jahr};
                \draw[thick, ->] (0,0) -- (0,3.8) node[above] {\tiny Schaden};
                
                % Balken
                \draw[fill=TechBlue!40, draw=none] (0.5,0) rectangle (1.2, 1.0) node[above, black, font=\tiny] {2020};
                \draw[fill=TechBlue!60, draw=none] (1.5,0) rectangle (2.2, 1.8) node[above, black, font=\tiny] {2022};
                \draw[fill=CyberCyan, draw=none] (2.5,0) rectangle (3.2, 3.2) node[above, black, font=\tiny] {2024};
                
                \node[align=center, font=\footnotesize, below=0.5cm of current bounding box] {Exponentieller Anstieg\\der Schäden};
            \end{tikzpicture}
    \end{columns}
\end{frame}

\begin{frame}{Die "Cyber Kill Chain" einer Attacke}
    Der typische Ablauf eines modernen Ransomware-Angriffs:
    
    \vspace{2em}
    \centering
    % TikZ Flowchart der Killchain
    \begin{tikzpicture}[
        node distance=0.8cm,
        phase/.style={rectangle, draw=TechBlue, thick, fill=white, rounded corners, minimum width=2.2cm, minimum height=1.2cm, align=center, font=\scriptsize, drop shadow},
        arrow/.style={->, >=stealth, thick, color=CyberCyan}
    ]
        \node[phase] (init) {\textbf{1. Initial}\\Access\\(Phishing/RDP)};
        \node[phase, right=of init] (c2) {\textbf{2. C2}\\Communication\\(Beaconing)};
        \node[phase, right=of c2] (lat) {\textbf{3. Lateral}\\Movement\\(Recon)};
        \node[phase, right=of lat, fill=CyberCyan!20] (enc) {\textbf{4. Action}\\Encryption\\(Impact)};
        
        \draw[arrow] (init) -- (c2);
        \draw[arrow] (c2) -- (lat);
        \draw[arrow] (lat) -- (enc);
    \end{tikzpicture}
    
    \vspace{2em}
    \raggedright
    \footnotesize
    \textbf{Hinweis:} In unserer Simulation fokussieren wir uns auf Schritt 2 (C2) und Schritt 4 (Encryption).
\end{frame}

% ==============================================================================
% TEIL 2: DIE TECHNISCHE ANALYSE (Simulation)
% ==============================================================================

\section{Analyse der Simulation}

\begin{frame}{Architektur der Simulation}
    \centering
    \begin{tikzpicture}[
        node distance=2cm,
        box/.style={rectangle, draw=none, fill=LightGrey, rounded corners, minimum height=1.5cm, minimum width=3.5cm, align=center, drop shadow},
        label/.style={font=\tiny, color=gray},
        arrow/.style={->, >=stealth, thick, color=TechBlue}
    ]
        % Nodes
        \node[box] (victim) {\textbf{\textcolor{CyberCyan}{RUST AGENT}}\\(Victim Client)};
        \node[box, right=4cm of victim] (c2) {\textbf{\textcolor{TechBlue}{PYTHON C2}}\\(Attacker Server)};
        
        % Firewall Wall Visualisierung
        \draw[line width=2pt, dashed, color=gray!50] (4, -1.8) -- (4, 1.8);
        \node[fill=white, text=gray, font=\tiny, inner sep=2pt] at (4,0) {Firewall / NAT};
        
        % Arrows
        \draw[arrow] ([yshift=8pt]victim.east) -- node[above, font=\tiny, align=center] {1. Reverse TCP Connect\\(Port 4444)} ([yshift=8pt]c2.west);
        \draw[arrow, dashed] ([yshift=-8pt]c2.west) -- node[below, font=\tiny, align=center] {2. Commands\\(encrypt/decrypt)} ([yshift=-8pt]victim.east);
        
    \end{tikzpicture}

    \vspace{1.5em}
    \begin{itemize}
        \item \textbf{Reverse Shell Prinzip:} Der Agent baut die Verbindung auf, nicht der Server. Das umgeht eingehende Firewall-Regeln.
        \item \textbf{Technologie:} Rust (Client) für Performance und Sicherheit, Python (Server) für einfache Handhabung.
    \end{itemize}
\end{frame}

\begin{frame}{Der Agent: Warum Rust?}
    Rust etabliert sich zunehmend in der Malware-Entwicklung ("Rustyware").
    
    \vspace{1em}
    
    \begin{columns}[T,onlytextwidth]
        \column{0.5\textwidth}
            \begin{block}{Technische Vorteile}
                \begin{itemize}
                    \item \textbf{Memory Safety:} Keine Buffer Overflows (weniger Abstürze beim Opfer).
                    \item \textbf{Zero Runtime:} Keine Installation von Python/Java nötig.
                    \item \textbf{Cross-Platform:} Ein Code für Linux, Windows \& macOS.
                \end{itemize}
            \end{block}
        
        \column{0.45\textwidth}
            \begin{block}{Evasion (Erkennung)}
                \begin{itemize}
                    \item Komplexer Maschinencode erschwert Reverse Engineering.
                    \item Geringere Erkennungsrate bei klassischen AV-Signaturen im Vergleich zu C/C++.
                \end{itemize}
            \end{block}
    \end{columns}
\end{frame}

% Cryptography Flow
\begin{frame}{Kryptographie: Der "Atomic Lock"}
    Ein kritischer Fehler bei Malware ist Datenverlust durch Abstürze. Unser Agent nutzt ein atomares Verfahren:

    \vspace{1.5em}
    \centering
    \begin{tikzpicture}[
        node distance=0.5cm,
        proc/.style={rectangle, draw=TechBlue, thick, fill=white, rounded corners, minimum width=2.2cm, minimum height=1.2cm, align=center, font=\scriptsize},
        arrow/.style={->, >=stealth, thick, color=CyberCyan}
    ]
        \node[proc] (read) {1. Stream Read\\(4KB Chunks)};
        \node[proc, right=of read] (enc) {2. AES-256-CTR\\(XOR Stream)};
        \node[proc, right=of enc] (write) {3. Write Temp\\\texttt{.enc\_temp}};
        \node[proc, right=of write, fill=TechBlue, text=white] (rename) {4. Atomic Rename\\\texttt{.locked}};

        \draw[arrow] (read) -- (enc);
        \draw[arrow] (enc) -- (write);
        \draw[arrow] (write) -- (rename);
        
        % Loop annotation
        \draw[->, gray, dashed] ($(write.south) + (-0.2, 0)$) to[out=240,in=300] node[below, font=\tiny] {Loop bis EOF} ($(read.south) + (0.2, 0)$);
    \end{tikzpicture}
    
    \vspace{1.5em}
    \raggedright
    \footnotesize
    \textbf{AES-256-CTR:} Der Counter-Mode ermöglicht schnelles Verschlüsseln ohne "Padding" (Dateigröße bleibt identisch).
\end{frame}

\begin{frame}[fragile]{Persistenz: Das Überleben des Neustarts}
    Malware muss sich im System verankern. Der Rust-Code unterscheidet zur Kompilierzeit das OS:
    
    \begin{columns}
        \column{0.5\textwidth}
            \textbf{Windows (Registry)}
            \begin{lstlisting}
Command::new("reg")
    .args([
        "add", 
        "HKCU\\...\\Run",
        "/v", "Updater",
        "/d", exe_path
    ])
            \end{lstlisting}
            
        \column{0.5\textwidth}
            \textbf{Linux (Systemd)}
            \begin{lstlisting}
[Unit]
Description=SystemSvc

[Service]
ExecStart=/path/to/agent
Restart=always
            \end{lstlisting}
    \end{columns}
\end{frame}

% ==============================================================================
% TEIL 3: DEMO WORKFLOW (Praxis)
% ==============================================================================

\section{Live Demo Workflow}

\begin{frame}{Ablauf der Demonstration}
    Wie die Komponenten im Test zusammenspielen:
    
    \vspace{1em}
    
    \begin{enumerate}
        \item \textbf{Setup:} Starten des Python C2 Servers (Listener auf Port 4444).
        \item \textbf{Infektion:} Ausführen der \texttt{rust-mw} Binary auf dem Zielsystem.
        \item \textbf{Handshake:} Client meldet sich beim Server ("New Session").
        \item \textbf{Angriff:} 
            \begin{itemize}
                \item Server sendet Befehl: \texttt{encrypt}
                \item Client generiert Key $\rightarrow$ Verschlüsselt Dateien $\rightarrow$ Zeigt Lösegeldforderung.
            \end{itemize}
        \item \textbf{Lösung:} Server sendet Befehl \texttt{decrypt} (simulierte Zahlung).
    \end{enumerate}
\end{frame}

% Conclusion
\begin{frame}[standout]
    \Huge Fazit
    
    \normalsize
    \vspace{2em}
    \begin{itemize}
        \item[\checkmark] \textbf{Simpel:} Sockets + Standard-Krypto reichen für maximalen Schaden.
        \item[\checkmark] \textbf{Effizient:} Rust ermöglicht extrem schnelle, schwer erkennbare Malware.
        \item[\checkmark] \textbf{Relevant:} Backup-Strategien sind der einzige wirkliche Schutz.
    \end{itemize}
\end{frame}

\end{document}
