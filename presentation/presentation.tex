\documentclass[aspectratio=169]{beamer}

% Theme und Packages
\usetheme{metropolis}
\usepackage[utf8]{inputenc}
\usepackage[T1]{fontenc}
\usepackage[ngerman]{babel}
\usepackage{graphicx}
\usepackage{booktabs}
\usepackage{xcolor}
\usepackage{tikz}
\usepackage{pgfplots}
\pgfplotsset{compat=1.18}
\usepackage[edges]{forest}

% Farbschema
\definecolor{darkaccent}{RGB}{35,55,75}
\definecolor{chartcolor1}{RGB}{231,76,60}
\definecolor{chartcolor2}{RGB}{52,152,219}
\definecolor{chartcolor3}{RGB}{46,204,113}
\definecolor{chartcolor4}{RGB}{241,196,15}
\definecolor{chartcolor5}{RGB}{155,89,182}
\definecolor{chartcolor6}{RGB}{230,126,34}

\setbeamercolor{frametitle}{bg=darkaccent}
\setbeamercolor{progress bar}{fg=orange,bg=gray}

% Titelseite Informationen
\title{Ransomware}
\subtitle{Anatomie, Netzwerkerkennung und Zero-Loss Recovery}
\author{Ihr Name}
\date{\today}
\institute{Ihre Institution}

\begin{document}

% Titelfolie
\begin{frame}
  \titlepage
\end{frame}

% Agenda
\begin{frame}{Agenda}
  \tableofcontents
\end{frame}

% Sektion 1: Entwicklung & Geschichte
\section{Entwicklung \& Geschichte}

\begin{frame}{Historischer Abriss}
  \begin{columns}
    \begin{column}{0.5\textwidth}
      \textbf{Meilensteine:}
      \begin{itemize}
        \item \textbf{1989}: AIDS Trojan - erster dokumentierter Ransomware-Angriff
        \item \textbf{2005}: Wiederaufleben durch Internet und Kryptowährungen
        \item \textbf{2017}: WannaCry-Ausbruch mit globaler Aufmerksamkeit
      \end{itemize}
    \end{column}
    \begin{column}{0.5\textwidth}
      \begin{figure}
        \includegraphics[width=\textwidth]{images/historical_timeline.png}
        \caption{Entwicklung von Ransomware über die Jahre}
      \end{figure}
    \end{column}
  \end{columns}
\end{frame}

% Sektion 2: Aktuelle Bedrohungslage
\section{Aktuelle Bedrohungslage}

\begin{frame}{Finanzielle Auswirkungen}
  \begin{columns}
    \begin{column}{0.45\textwidth}
      \textbf{Geschätzte Weltweite Schäden}
      \begin{itemize}
        \item 2017: 5 Mrd. USD
        \item 2024: 42 Mrd. USD
      \end{itemize}
      \vspace{0.5em}
    \end{column}
    \begin{column}{0.55\textwidth}
      \begin{tikzpicture}
        \begin{axis}[
          width=\textwidth,
          height=0.7\textheight,
          xlabel={Jahr},
          ylabel={Schaden (Mrd. USD)},
          ymin=0, ymax=45,
          xtick={2015,2017,2019,2021,2024},
          ytick={0,10,20,30,40},
          grid=major,
          legend pos=north west,
          ylabel near ticks,
          xlabel near ticks,
          x tick label style={/pgf/number format/1000 sep={}}
        ]
        \addplot[
          color=chartcolor1,
          mark=*,
          line width=2pt
        ] coordinates {
          (2015,0.325)
          (2017,5)
          (2019,11.5)
          (2021,20)
          (2024,42)
        };
        \legend{Weltweite Schäden}
        \end{axis}
      \end{tikzpicture}
    \end{column}
  \end{columns}
\end{frame}

\begin{frame}{ENISA Threat Landscape 2024}
  \begin{columns}
    \begin{column}{0.4\textwidth}
      \textbf{Top-Bedrohungen EU (Prime Threats):}
      \begin{enumerate}
        \item DDoS/RDoS (46,31\%)
        \item \alert{Ransomware (27,33\%)}
        \item Data Breaches (15,87\%)
        \item Social Engineering (5,37\%)
        \item Malware (2,45\%)
        \item Zero-Day (0,11\%)
      \end{enumerate}
      \vspace{0.5em}
      Ransomware: Zweitgrö{\ss}te Bedrohung
    \end{column}
    \begin{column}{0.6\textwidth}
      \begin{tikzpicture}
        \begin{axis}[
          width=0.92\textwidth,
          height=0.75\textheight,
          xbar,
          xlabel={Vorfälle (in Tausend)},
          symbolic y coords={Zero-Day, Malware, Social Eng., Data, Ransomware, DDoS},
          ytick=data,
          nodes near coords,
          nodes near coords align={horizontal},
          xmin=0, xmax=2.8,
          bar width=0.5cm,
          ylabel near ticks,
          xlabel near ticks
        ]
        \addplot[fill=chartcolor1] coordinates {
          (0.01,Zero-Day)
          (0.13,Malware)
          (0.29,Social Eng.)
          (0.84,Data)
          (1.45,Ransomware)
          (2.46,DDoS)
        };
        \end{axis}
      \end{tikzpicture}
    \end{column}
  \end{columns}
\end{frame}

% Sektion 3: Ziele & Motivation
\section{Ziele \& Motivation}

\begin{frame}{Am häufigsten betroffene Sektoren}
  \begin{columns}
    \begin{column}{0.38\textwidth}
      \textbf{Top 5 Sektoren:}
      \begin{enumerate}
        \item Manufacturing (21,78\%)
        \item Retail (11,06\%)
        \item ICT Services (9,18\%)
        \item Business Services (7,14\%)
        \item Banking/Finance (6,94\%)
      \end{enumerate}
      \vspace{0.5em}
      \small{Industrie und Fertigung am stärksten betroffen}
    \end{column}
    \begin{column}{0.62\textwidth}
      \vspace{0.2cm}
      \begin{tikzpicture}
        \begin{axis}[
          width=0.88\textwidth,
          height=0.9\textheight,
          xbar,
          xlabel={Anteil (\%)},
          symbolic y coords={Water, Energy, Public Admin, Transport, Education, Banking, Business, ICT, Other, Retail, Manufacturing},
          ytick=data,
          nodes near coords,
          nodes near coords align={horizontal},
          xmin=0, xmax=24,
          bar width=0.35cm,
          ylabel near ticks,
          xlabel near ticks,
          every node near coord/.append style={font=\tiny}
        ]
        \addplot[fill=chartcolor2] coordinates {
          (0.37,Water)
          (2.78,Energy)
          (3.58,Public Admin)
          (3.97,Transport)
          (4.92,Education)
          (6.94,Banking)
          (7.14,Business)
          (9.18,ICT)
          (10.52,Other)
          (11.06,Retail)
          (21.78,Manufacturing)
        };
        \end{axis}
      \end{tikzpicture}
    \end{column}
  \end{columns}
\end{frame}

\begin{frame}{Motivation der Angreifer}
  \textbf{Warum diese Ziele?}
  \begin{itemize}
    \item \textbf{Finanzielle Motive}: Zahlungsbereitschaft bei kritischen Systemen
    \item \textbf{Kritische Infrastrukturen}: Hoher Druck durch Ausfallkosten
    \item \textbf{Sensible Daten}: Erpressungspotenzial durch Datenleaks
    \item \textbf{Geopolitische Faktoren}: Destabilisierung und Spionage
  \end{itemize}
  
  \vspace{0.5em}
  
  \alert{Zunehmende Professionalisierung und Organisation}
\end{frame}

% Sektion 4: Arten von Ransomware
\section{Arten von Ransomware}

\begin{frame}{Ransomware Taxonomy}
  % Wir definieren einen gemeinsamen Stil für beide Bäume
  \tikzset{
    mybox/.style={
      font=\scriptsize\sffamily\bfseries,
      draw=gray, line width=0.8pt, inner sep=3pt, rounded corners=2pt, align=center
    },
    greenbox/.style={mybox, fill=green!30!gray!40, text width=2.2cm, minimum height=1cm},
    bluebox/.style={mybox, rectangle, rounded corners=0, fill=cyan!20, text width=2.2cm, font=\scriptsize\sffamily}
  }

  % Zeile 1: Die ersten 3 Kategorien
  \centering
  \resizebox{0.95\textwidth}{!}{%
    \begin{forest}
      for tree={edge={draw, gray}, s sep=0.3cm, l sep=0.5cm}
      [, phantom, forked edges % Unsichtbare Wurzel
        [Types, greenbox, for children={folder, grow'=0} [Crypto, bluebox][Locker, bluebox][Scareware, bluebox]]
        [Targets, greenbox, for children={folder, grow'=0} [Platforms, bluebox][Industries, bluebox]]
        [Infection vectors, greenbox, for children={folder, grow'=0} [Windows RDP, bluebox][Phishing, bluebox][Software vuln., bluebox]]
      ]
    \end{forest}
  }

  \vspace{0.5cm} % Abstand zwischen den Zeilen

  % Zeile 2: Die restlichen 3 Kategorien
  \resizebox{0.95\textwidth}{!}{%
    \begin{forest}
      for tree={edge={draw, gray}, s sep=0.3cm, l sep=0.5cm}
      [, phantom, forked edges
        [Comm. with C\&C, greenbox, for children={folder, grow'=0} [Hard-coded IP, bluebox][DGA, bluebox]]
        [Access to enc. key, greenbox, for children={folder, grow'=0} [Hard-coded binary, bluebox][Via C\&C, bluebox]]
        [Malicious action, greenbox, for children={folder, grow'=0} [Encryption, bluebox][Locking, bluebox]]
      ]
    \end{forest}
  }
\end{frame}

\begin{frame}{Grundlegende Typen}
  \begin{columns}
    \begin{column}{0.5\textwidth}
      \textbf{1. Crypto-Ransomware}
      \begin{itemize}
        \item Verschlüsselt Benutzerdateien
        \item Verwendet AES, RSA
        \item System bleibt funktional
        \item Wiederherstellung oft unmöglich
      \end{itemize}
    \end{column}
    \begin{column}{0.5\textwidth}
      \textbf{2. Locker-Ransomware}
      \begin{itemize}
        \item Sperrt Systemzugriff
        \item Bildschirmsperre
        \item Keine Datenverschlüsselung
        \item Einfacher zu beheben
      \end{itemize}
    \end{column}
  \end{columns}
\end{frame}

\begin{frame}{Moderne Erpressungsformen}
  \textbf{Evolution der Angriffsmethoden:}
  
  \begin{itemize}
    \item \textbf{Leakware/Doxware}: Drohung mit Veröffentlichung sensibler Daten
    
    \item \textbf{Double Extortion}: 
    \begin{itemize}
      \item Verschlüsselung \textit{und} Datenexfiltration
      \item Doppelter Zahlungsdruck
    \end{itemize}
    
    \item \textbf{Triple Extortion}:
    \begin{itemize}
      \item Zusätzliche Drohung gegen Kunden/Partner
      \item DDoS-Angriffe als dritte Ebene
    \end{itemize}
    
    \item \textbf{Ransomware-as-a-Service (RaaS)}:
    \begin{itemize}
      \item Mietmodell für Cyberkriminelle
      \item Keine technischen Kenntnisse erforderlich
      \item Demokratisierung der Cyberkriminalität
    \end{itemize}
  \end{itemize}
\end{frame}

\begin{frame}{Wichtige RaaS-Varianten}
  \begin{description}
    \item[LockBit]: Globale Dominanz, schnelle Verschlüsselung, hochprofessionell
    \item[Maze]: Pionier der Double-Extortion-Methode
    \item[Ryuk]: Gezielter gegen Gro{\ss}unternehmen
  \end{description}
\end{frame}

% Sektion 5: Die Phasen eines Angriffs
\section{Die Phasen eines Angriffs}

\begin{frame}{Ransomware Kill Chain: Überblick}
 %  
 %  \begin{enumerate}
 %    \item \textbf{Infection (Infektion)}: Ransomware wird auf das Opfersystem geliefert (Phishing, Software-Schwachstellen).
 %    \item \textbf{C\&C/C2 Communiaction}: Nach der Infektion kontaktiert die Malware den Command-and-Control Server um Informationen auszutauschen.
 %    \item \textbf{Destruction Zerstörung}: Die eigentliche bösartige Aktion wird durchgeführt (Verschlüsselung, Sperrung des Systemzugriffs).
 %    \item \textbf{Extortion (Erpressung)}: Dem Opfer wird eine Lösegeldforderung gezeigt, die Anweisungen für die Zahlung enthält (typischerw. Kryptowährungen).


 %  \end{enumerate}
 %  
 %  \vspace{0.5em}
 %  
  \begin{figure}
    \includegraphics[width=\textwidth]{images/attack_phases_overview.png}
    \caption{Die vier Phasen eines Ransomware-Angriffs}
  \end{figure}
\end{frame}

\begin{frame}{Phase 1-2: Distribution \& Infection}
  \begin{columns}
    \begin{column}{0.5\textwidth}
      \textbf{Distribution - Angriffsvektoren:}
      \begin{itemize}
        \item \textbf{Phishing-E-Mails} (häufigster Vektor)
        \begin{itemize}
          \item Bösartige Anhänge
          \item Manipulierte Links
        \end{itemize}
        \item Remote Desktop Protocol (RDP)
        \begin{itemize}
          \item Brute-Force-Angriffe
          \item Gestohlene Credentials
        \end{itemize}
        \item Software-Schwachstellen
        \begin{itemize}
          \item Ungepatchte Systeme
          \item Zero-Day-Exploits
        \end{itemize}
      \end{itemize}
    \end{column}
    \begin{column}{0.5\textwidth}
      \textbf{Infection:}
      \begin{itemize}
        \item Installation der Malware
        \item Ausführung des Schadcodes
        \item Umgehung von Sicherheitsmechanismen
        \item Moderne Techniken:
        \begin{itemize}
          \item Rootkit-Technologien
          \item Verzögerte Ausführung
        \end{itemize}
      \end{itemize}
    \end{column}
  \end{columns}
\end{frame}

\begin{frame}{Phase 3-4: Staging \& Scanning}
  \textbf{Staging - Etablierung im System:}
  \begin{itemize}
    \item Persistenzmechanismen (Registry, Autostart)
    \item Aufbau der Command \& Control (C2) Kommunikation
    \item Lateral Movement im Netzwerk
    \item Rechteerweiterung (Privilege Escalation)
  \end{itemize}
  
  \vspace{0.5em}
  
  \textbf{Scanning - Zielerkennung:}
  \begin{itemize}
    \item Durchsuchen lokaler Laufwerke
    \item Identifikation von Netzwerkfreigaben
    \item Cloud-Speicher-Erkennung
    \item Priorisierung wertvoller Daten
  \end{itemize}
  
\end{frame}

\begin{frame}{Phase 5-6: Encryption \& Payment}
    \textbf{Encryption:}
    \begin{itemize}
      \item Starke Verschlüsselung (AES, RSA)
      \item Hybride Verschlüsselungsverfahren
      \item Schnelle Verarbeitung
      \item Löschung von Backups/Schattenkopien
    \end{itemize}
      
      \vspace{0.5em}
      
    \textbf{Payment:}
    \begin{itemize}
      \item Anzeige der Lösegeldforderung
      \item Zahlungsanweisungen (Kryptowährung)
      \item Zeitlimit mit Preiserhöhung
      \item Drohung mit Datenlöschung/-veröffentlichung
    \end{itemize}
\end{frame}

\begin{frame}{Moderne Entwicklungen in Angriffstechniken}
  \textbf{KI-gestützte Angriffe:}
  \begin{itemize}
    \item KI-generierte Phishing-Kampagnen
    \item Umgehung von ML-basierten Detektionssystemen
    \item Automatisierte Zielerkennung und -priorisierung
  \end{itemize}
  
  \vspace{0.5em}
  
  \textbf{Weitere Trends:}
  \begin{itemize}
    \item Rootkit-Fashion: Tiefere Systemintegration
    \item Wiper-Ransomware: Zerstörung statt Verschlüsselung
    \item Supply-Chain-Angriffe über vertrauenswürdige Software
  \end{itemize}
  
  \vspace{0.5em}
  
  \begin{figure}
    % \includegraphics[width=0.7\textwidth]{modern_techniques}
    \caption{Evolution der Angriffstechniken}
  \end{figure}
\end{frame}

% Sektion 6: Gegenmaßnahmen
\section{Gegenma{\ss}nahmen}

\begin{frame}{Prävention}
  \textbf{Technische Ma{\ss}nahmen:}
  \begin{itemize}
    \item \textbf{Backups}: 3-2-1-Regel (3 Kopien, 2 Medien, 1 Offsite)
    \item \textbf{Patch-Management}: Regelmä{\ss}ige Updates und Schwachstellenbeseitigung
    \item \textbf{Netzwerksegmentierung}: Begrenzung der lateralen Bewegung
    \item \textbf{Access Control}: Least Privilege Prinzip, MFA
  \end{itemize}
  
  \vspace{0.5em}
  
  \textbf{Organisatorische Ma{\ss}nahmen:}
  \begin{itemize}
    \item Security Awareness Training
    \item Incident Response Pläne
    \item Regelmä{\ss}ige Sicherheitsaudits
  \end{itemize}
\end{frame}

\begin{frame}{Detektion und Reaktion}
  \textbf{Detektionsansätze:}
  \begin{itemize}
    \item \textbf{Statische Analyse}: Untersuchung ohne Ausführung
    \item \textbf{Dynamische Analyse}: Verhaltensbeobachtung in Sandbox
    \item \textbf{Machine Learning}: Erkennung unbekannter Varianten
    \item \textbf{Netzwerkbasierte Detektion}: Anomalie-Erkennung im Traffic
  \end{itemize}
  
  \vspace{0.5em}
  
  \textbf{Wichtige Merkmale:}
  \begin{itemize}
    \item API-Aufrufe und Systemverhalten
    \item Datei-/Verzeichnisaktivitäten
    \item Netzwerkverkehrsmuster
    \item Verschlüsselungsoperationen
  \end{itemize}
\end{frame}

\begin{frame}{Incident Response}
  \textbf{Sofortma{\ss}nahmen bei Verdacht:}
  \begin{enumerate}
    \item \alert{Isolation} betroffener Systeme
    \item Identifikation des Ransomware-Stamms
    \item Bewertung der Schadenausbreitung
    \item Benachrichtigung relevanter Stellen
  \end{enumerate}
  
  \vspace{0.3em}
  
  \textbf{Wiederherstellung:}
  \begin{itemize}
    \item Wiederherstellung aus Backups (wenn möglich)
    \item Key-Escrow-Mechanismen
    \item Forensische Analyse
    \item Systemhärtung vor Wiederinbetriebnahme
  \end{itemize}
  
  \vspace{0.5em}
  
  \alert{Zahlungsempfehlung}: Keine Lösegeldzahlung (keine Garantie, finanziert weitere Angriffe)
\end{frame}

% Sektion 7: Forschung & Ausblick
\section{Forschung \& Ausblick}

\begin{frame}{Aktuelle Forschungslandschaft}
  \textbf{Fokusgebiete:}
  \begin{itemize}
    \item \textbf{72,8\%} der Studien: Ransomware-Detektion
    \item Machine Learning dominiert als Detektionsmethode
    \item Hohe Genauigkeit bei unbekannten Varianten
  \end{itemize}
  
  \vspace{0.5em}
  
  \textbf{Innovative Ansätze:}
  \begin{itemize}
    \item \textbf{REDFISH}: Netzwerkbasierte Früherkennung
    \begin{itemize}
      \item Monitoring von SMB-Protokoll
      \item Erkennung von Verhaltensmustern (schnelles Lesen/Schreiben/Löschen)
      \item \alert{99\% Erkennung vor Verlust von 10 Dateien}
    \end{itemize}
    \item Near-Zero-Loss-Szenario durch Traffic-Wiederherstellung
    \item Moving Target Defense (MTD)
  \end{itemize}
\end{frame}

\begin{frame}{Zero-Loss Recovery: REDFISH}
  \begin{columns}
    \begin{column}{0.5\textwidth}
      \textbf{Funktionsweise:}
      \begin{itemize}
        \item Analyse des Netzwerk-Traffics zu Freigaben
        \item Erkennung anomaler Zugriffsmuster
        \item Aufzeichnung der Dateiinhalte
        \item Automatische Alarmierung
      \end{itemize}
      
      \vspace{0.5em}
      
      \textbf{Vorteile:}
      \begin{itemize}
        \item Früherkennung (99\% bei <10 Dateien)
        \item Wiederherstellung ohne Backups
        \item Minimaler Datenverlust
      \end{itemize}
    \end{column}
    \begin{column}{0.5\textwidth}
      \begin{figure}
        % \includegraphics[width=\textwidth]{redfish_system}
        \caption{REDFISH Architektur}
      \end{figure}
      
      \vspace{0.5em}
      
      \begin{figure}
        % \includegraphics[width=\textwidth]{detection_rate}
        \caption{Erkennungsrate vs. Datenverlust}
      \end{figure}
    \end{column}
  \end{columns}
\end{frame}

\begin{frame}{Zukünftige Herausforderungen}
  \textbf{Technologische Entwicklungen:}
  \begin{itemize}
    \item \textbf{KI in Ransomware}: Adaptives, intelligentes Verhalten
    \item \textbf{Quantencomputing}: Bedrohung für aktuelle Verschlüsselung
    \item \textbf{IoT-Ransomware}: Neue Angriffsflächen
    \item \textbf{Cloud-native Ransomware}: Angriffe auf Cloud-Infrastrukturen
  \end{itemize}
  
  \vspace{0.5em}
  
  \textbf{Forschungsbedarf:}
  \begin{itemize}
    \item Echtzeit-Schutz und Zero-Day-Erkennung
    \item Post-Quantum-Kryptographie
    \item Automatisierte Incident Response
    \item Internationale Zusammenarbeit bei Strafverfolgung
  \end{itemize}
\end{frame}

% Sektion 8: Fazit
\section{Fazit}

\begin{frame}{Zusammenfassung}
  \textbf{Kernaussagen:}
  \begin{itemize}
    \item Ransomware bleibt eine der grö{\ss}ten Cyberbedrohungen
    \item Kontinuierliche Evolution der Angriffsmethoden (RaaS, Multi-Extortion)
    \item Kritische Infrastrukturen besonders gefährdet
    \item \alert{Prävention ist der beste Schutz}
  \end{itemize}
  
  \vspace{0.5em}
  
  \textbf{Handlungsempfehlungen:}
  \begin{itemize}
    \item Mehrschichtige Sicherheitsarchitektur
    \item Regelmä{\ss}ige Backups und Tests
    \item Kontinuierliche Schulung der Mitarbeiter
    \item Investition in moderne Detektionssysteme
    \item Vorbereitung von Incident Response Plänen
  \end{itemize}
\end{frame}

\begin{frame}{Quellen}
  \begin{thebibliography}{99}
    \bibitem{age} \textit{The Age of Ransomware: A Survey on the Evolution, Taxonomy, and Research Directions} (Razulla et al.)
    \bibitem{enisa} ENISA (2024): \textit{Threat Landscape Report 2024} (European Union Agency for Cybersecurity)
    \bibitem{redfish} \textit{Ransomware early detection by the analysis of file sharing traffic (REDFISH)} (Morato et al.)
    \bibitem{presentation} \textit{A Survey on Ransomware: Evolution, Taxonomy, and Defense Solutions} (Oz et al.)
  \end{thebibliography}
\end{frame}

\end{document}
